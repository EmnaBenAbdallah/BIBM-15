The paper is motivated by the problem of steady states identification as well as reachability problem in biological regulatory systems that describes genes and proteins interactions. Indeed the regulatory phenomena play a crucial role in biological systems, so they need to be studied accurately. Thus the study of Biological Regulatory Networks (BRNs) has been the subject of numerous researches \cite{thieffry1999modularity, shermin20092, rauf2011formal}. BRNs consist in sets of either positive or negative mutual effects between the components. With the purpose of analyzing these systems, they are often modeled as graphs which makes it possible to determine the possible evolutions of all the interacting components of the system. Thus, in order to address the formal checking of dynamical properties within very large BRNs, new formalisms and conversely new techniques have been proposed during the last decade. For example, Boolean networks \cite{stuart1993origins, kauffman1969metabolic} have been used due to their simplicity and ability to handle noisy data. But this binary representation of genes in the network is a way to lose data information. %Besides, artificial neural networks omit using a hidden layer so that they can be interpreted, losing the ability to model higher order correlations in the data.

We can cite some works that manipulate these models and which have been developed to verify the dynamic properties of BRNs modeled: \textsc{GINsim} (Gene Interaction Network simulation) \cite{chaouiya2012logical, gonzalez2006ginsim} is devoted to the modeling and simulation of genetic regulatory networks, based on a logical approach. This tool is very convenient  to define a logical regulatory graph through a dedicated graphical interface. However it is not always possible to manipulate large-scale networks. Indeed, in order to verify a dynamical property, such as reachability, the tool computes the whole Boolean network from the corresponding Thomas network.
We can also cite \textsc{libDDD} (Library of Data Decision Diagrams) a library for symbolic model-checking of CTL \& LTL properties \cite{libddd, Kordon09libddd}. It can especially be used to check reachability properties, but it doest not output the execution path solving the reachability. The computing tests (see section \pref{sec:comparison}) show that \textsc{LibDDD} and \textsc{GINsim} take a long time to respond, and get out of memory for the big examples we proposed (more than 40 components).  %à revoir
Several other promising modeling and analyzing techniques have been developed, using Boolean networks, Petri nets \cite{heljanko2001answer}, Bayesian networks \cite{numata2008partial} trying to optimise the computational time as well as the result. %, we can cite {ref des autres travaux sur l’atteignabilité} \ref{paoletti2014analyzing}

It is also common to model biological networks with a set of coupled ordinary differential equations (ODEs) \cite{chu2009models}, describing the kinetic reactions. But the temporal evolution of the systems modeled by ODEs is computed by a complex derivation approach. This complexity of computing the evolution makes that the verification of dynamic properties takes a long time to respond. 

The algorithms and practice of simulation of large systems with discrete events, formalisms specifying discrete event models have now reached a maturity that allows us to consider them as the equivalent formalisms models of the continuous differential equations systems This formalism allows, in addition to its traditional use in the computer world, to model dynamic systems (such BRNs) for their study or with the aim of controlling them \cite{chaouiya2008qualitative, balov2012discrete} .

In order to address the formal checking of dynamical properties within very large BRNs, a new discret formalism, named Process Hitting (PH), \cite{PMR10-TCSB}, to model concurrent systems having components with a few qualitative levels. A PH describes, in an atomic manner, the possible evolutions of a "process'' (representing one component at one level) triggered by the hit of other "processes'' in the system. Comparing with other models of BRNs, the particular structure of the PH makes the formal analysis of BRNs with hundreds of components tractable. This was proved by a first work on the PH in \cite{PMR12-MSCS} which analyzes big networks and gives a response in a very short time. But this developed technique was based on abstract methods computing approximations of the dynamics that could be inconclusive in some cases. Moreover, in the case of a positive answer, it currently does not return the execution of the path achieving the desired reachability, but only outputs its conclusion.

Our goal in this paper is to develop exhaustive methods to analyze Biological Regulatory Networks modeled in Process Hitting. With respect to PH dynamic, this analysis consists into three kinds of results:
\begin{itemize}
\item[-] Finding all possible steady states of a BRN,
\item[-] Simulating the evolution of a biological network,
\item[-] Computing the shortest execution path to reach a property.
\end{itemize}

 The particularity of our contribution relies in the use of Answer Set Programming
(ASP) \cite{baral2003knowledge}
to compute the results.
This declarative programming framework has been proven efficient
to tackle models with a large number of components and parameters.
Our aim here is to assess its potential w.r.t.\ the computation
of some dynamical properties of PH models. In addition, we chose the PH framework that was mainly used for biological networks. Besides, it allows to represent any kind of dynamical model,
and converters to several other representations are available and included into \textsc{Pint}\footnote{\textsc{Pint} is available at \url{http://loicpauleve.name/pint/}; version 2015-11-14 was used for this work.}~\cite{PMR12-MSCS}.
Indeed, the the particular form of the actions of the PH allows
to easily represent them using the Answer Set Programming \cite{Baral03, Vladimir, Glimpse, sureshkumar2006ansprolog},
with one fact per action.
We note that although an efficient dynamical analysis already exists for this framework,
provided with \textsc{Pint} and based on an approximation of the dynamics,
it is interesting to identify its limits
and compare them to the exhaustive approach we present in this paper.