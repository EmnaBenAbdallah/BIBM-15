This paper is motivated by the two specific -- although widespread -- problems of steady states identification and reachability checking in Biological Regulatory Networks (BRNs) that describe genes and proteins interactions. Indeed, the regulatory phenomena play a crucial role in biological systems, and they need to be studied accurately. The study of BRNs has consequently been the subject of numerous researches \cite{thieffry1999modularity, shermin20092}. BRNs consist in sets of either positive or negative mutual interactions between the components. With the purpose of analyzing these systems, they are often modeled as graphs which makes it possible to determine the possible evolutions of all the interacting components of the system. Thus, in order to address the formal checking of dynamical properties within very large BRNs, new formalisms and conversely new techniques have been proposed during the last decade. For example, Boolean networks \cite{stuart1993origins} have been widely used and studied due to their simplicity and ability to handle noisy data. But multi-valued discrete networks are now preferred due to their higher modeling power.

%But this binary representation of genes in the network is a way to lose data information. %Besides, artificial neural networks omit using a hidden layer so that they can be interpreted, losing the ability to model higher order correlations in the data.
The original aim of discrete networks \cite{kauffman69,Thomas73} was to bypass the complexity of
continuous differential equation-based modeling.
However, these discrete representations quickly gain in complexity -- in order to represent more complex behaviors or dynamical constraints --
and in popularity -- leading to the creation of models with thousands of components, or even dedicated databases.
As more complex models arise, the need to study them is also confronted to the increasing complexity of such analyses
that are usually exponential with classical model checking methods \cite{Harel02}.
%We can cite some works that manipulate these models and which have been developed to verify the dynamic properties of BRNs: \textsc{GINsim} (Gene Interaction Network simulation) \cite{chaouiya2012logical, gonzalez2006ginsim} is devoted to the modeling and simulation of genetic regulatory networks, based on a logical approach. This tool is very convenient  to define a logical regulatory graph through a dedicated graphical interface. However it is not always possible to manipulate large-scale networks.
%We can also cite the library \textsc{libDDD} (Library of Data Decision Diagrams) for symbolic model-checking of CTL \& LTL properties \cite{libddd, Kordon09libddd}. It can especially be used to check reachability properties on Petri nets.
%However, these tools compute the whole state-transition graph corresponding to a BRN, which requires exponential space,
%and thus are not always able to check dynamical properties on large networks.
%Several other promising modeling and analyzing techniques have been developed, using Boolean networks, Petri nets \cite{heljanko2001answer} or Bayesian networks \cite{numata2008partial} by trying to optimize the computational time as well as the result. %, we can cite {ref des autres travaux sur l’atteignabilité} \ref{paoletti2014analyzing}

% It is also common to model biological networks with a set of coupled ordinary differential equations (ODEs) \cite{chu2009models}, describing the kinetic reactions. But the temporal evolution of the systems modeled by ODEs is computed by a complex derivation approach. This complexity of computing the evolution makes that the verification of dynamic properties takes a long time to respond. 

% Studying the dynamics of large systems with discrete events
% has reached a maturity that makes it catch up with the complexity of continuous differential equations systems.
% Such formalisms allow, in addition to their traditional use in the computer world,
% to model dynamic systems (such as BRNs) in order to study their behavior or with the aim to control them \cite{chaouiya2008qualitative, balov2012discrete}.
In order to address the formal checking of dynamical properties within very large BRNs, a new discrete formalism, named Process Hitting (PH) \cite{PMR10-TCSB}, was recently proposed to model concurrent systems having components with few qualitative levels. A PH describes, in an atomic manner, the possible evolutions of a ``process'' (representing one component at one level) triggered by the hit of other ``processes'' in the system. Compared with other BRNs formalisms, the particular structure of the PH makes the formal analysis of BRNs with hundreds of components tractable~\cite{PMR12-MSCS}. 
% A voir à laisser ou à enlever
%This was proved by a first work on the PH in \cite{PMR12-MSCS} which analyzes big networks and gives a response in a very short time. But this developed technique was based on abstract methods computing approximations of the dynamics that could be inconclusive in some cases. Moreover, in the case of a positive answer, it currently does not return the execution of the path achieving the desired reachability, but only outputs its conclusion.

Our goal in this paper is to develop exhaustive methods to analyze Biological Regulatory Networks modeled in Process Hitting. With respect to PH dynamics, this analysis consists in three kinds of results:
\begin{itemize}
\item[-] Finding all possible steady states of a BRN,
\item[-] Simulating the evolution of a biological network,
\item[-] Computing the shortest execution path to reach a goal.
\end{itemize}
 The particularity of our contribution relies in the use of Answer Set Programming
(ASP) \cite{baral2003knowledge}
to compute the results.
This declarative programming framework has proved efficient
to tackle models with a large number of components and parameters.
Our aim here is to assess its potential w.r.t.\ the computation
of some dynamical properties of PH models.
We chose the PH framework because it allows to represent a wide range of dynamical models,
%and converters to several other representations are available and included into \textsc{Pint}\footnote{\textsc{Pint} version 2015-11-14 has been used for this work; it is available at: \url{http://loicpauleve.name/pint/}}~\cite{PMR12-MSCS}.
and the particular form of its actions
can be easily represented using ASP,
with exactly one fact per action.
% A voir à laisser ou à enlever
%We note that although an efficient dynamical analysis already exists for this framework,
%provided with \textsc{Pint} and based on an approximation of the dynamics,
%it is interesting to identify its limits
%and compare them to the exhaustive approach we present in this paper.