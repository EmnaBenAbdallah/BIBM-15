The paper is motivated by the problem of steady states identification as well as reachability problem in biological regulatory systems that describes genes and proteins interactions. Indeed the regulatory phenomena play a crucial role in biological systems, so they need to be studied accurately. Thus the study of Biological Regulatory Networks (BRNs) has been the subject of numerous researches \cite{thieffry1999modularity, shermin20092, rauf2011formal}. BRNs consist in sets of either positive or negative mutual effects between the components. With the purpose of analyzing these systems, they are often modeled as graphs which makes it possible to determine the possible evolutions of all the interacting components of the system. Thus, in order to address the formal checking of dynamical properties within very large BRNs, new formalisms and conversely new techniques have been proposed during the last decade. For example, Boolean networks \cite{stuart1993origins, kauffman1969metabolic} have been used due to their simplicity and ability to handle noisy data. But this binary representation of genes in the network is a way to lose data information. Besides, artificial neural networks omit using a hidden layer so that they can be interpreted, losing the ability to model higher order correlations in the data. We can cite some works that are manipulating with these models and which have been developed to verify the dynamic properties of BRNs modeled: \textsc{GINsim} (Gene Interaction Network simulation) \cite{chaouiya2012logical, gonzalez2006ginsim} is devoted to the modelling and simulation of genetic regulatory networks, based on the logical approach. This tool is very convinient  to define a logical regulatory graph through a dedicated interface. However the inconvinient is when the user starts to manipulate large-scale networks. Indeed, in order to verify a dynamical property, the reachability, it begins by computing the whole boolean network from the corresponding Thomas network. \\ 
We can also cite \textsc{libDDD} (Library of Data Decision Diagrams) a library for symbolic model-checking of CTL \& LTL properties \cite{libddd, Kordon09libddd}. So it can especially be used to check reachability properties. But it cannot output the execution path solving the reachability. The computing tests (see section \pref{sec:comparison}) show that \textsc{LibDDD} and \textsc{GINsim} take a long time to respond, and gets out of memory for the big examples (more than 40 components).  %à revoir
Several other promising modeling and analysing techniques have been developed, using Boolean networks, Petri nets, Bayesian networks \cite{numata2008partial} trying to optimise the computational time as well as the result.\\ %, we can cite {ref des autres travaux sur l’atteignabilité} \ref{paoletti2014analyzing}
It is also common to model biological networks with a set of coupled ordinary differential equations (ODEs) \cite{chu2009models}, describing the kinetic reactions. But the temporal evolution of the systems modeled by ODEs is computed by a complex derivation approach. This complexity of computing the evolution so that the verification of the dynamic properties causes also a long time to respond. 

New experimental results \cite{elowitz2002stochastic} \cite{blake2003noise} demonstrated that gene expression is a stochastic process. Thus, many biologists and bioinformaticians are now using the stochastic formalism \cite{arkin1998stochastic, tian2006stochastic, wang2010robust}. This formalism permits to represent each gene expression \cite{raser2005noise} and small synthetic genetic networks, \cite{elowitz2000synthetic} \cite{gardner2000construction}.

 Such that our new formalism named Process Hitting (PH) \cite{PMR10-TCSB}. In order to address the formal checking of dynamical properties within very large BRNs, we recently introduced this new formalism (PH) \cite{PMR10-TCSB}, to model concurrent systems having components with a few qualitative levels. A PH describes, in an atomic manner, the possible evolutions of a "process'' (representing one component at one level) triggered by the hit of other "processes’’ in the system. Comparing with other models of BRNs, this particular structure of PH makes the formal analysis of BRNs with hundreds of components easier to be tractable. This was proved by a first work on the PH in \cite{PMR10-TCSB,PMR12-MSCS} which analysis big networks and gives a response through a short time. But this developed technic was based on abstract methods computing approximations of the dynamics that could be inconclusive in some cases. Moreover, in the case of a positive answer, it currently does not return the execution of the path achieving the desired reachability, but only outputs its conclusion.

Our goal in this paper is to develop exhaustive methods to analyze Biological Regulatory Networks modeled in Process Hitting. Whith respect to PH dynamic, this analyse consists into:
\begin{itemize}
\item[-] Simulating the evolution of the networks,
\item[-] Finding all possible steady states of BRNs in short time, and
\item[-] Computing the shorter execution path for the reachability property.
\end{itemize}

 The particularity of our contribution relies in the use of Answer Set Programming
(ASP) \cite{baral2003knowledge}
to compute the result of theis inference.
This declarative programming framework has been proven efficient
to tackle models with a large number of components and parameters.
Our aim here is to assess its potential w.r.t.\ the computation
of some dynamical properties of PH models.

The paper is organised as follows. \ref{sec:prem-def} definition of the PH framework and the dynamical properties. % à remplir
We conclude the paper with a short discussion. 