The study of basins of attraction provides an important understanding of the different behaviors of a Biological Regulatory Network (BRN)~\cite{wuensche1998genomic}.
Indeed, a system will always eventually end in a basin of attraction,
and this may depend on biological switches or other complex phenomena.
Here we focus on fixed points (also called stable states or steady states),
which are basins of attraction composed of only one state.
%In the following, we consider a Process Hitting $\PH = (\Sigma, \PHl, \PHh)$.
%We recall that a state $s \in L$ is a fixed point of the PH model if and only if it has no successor state,
%\ie there is no playable action in $s$ (see \pref{def:fixpoint}).
%It is straightforward that
%a stable state in a Process Hitting model is a state so that
%every active process does not hit or is not hit by another active process in the same state~\cite{PMR10-TCSB}.
%We note especially that given this result, processes involved in a self-hit (an action whose hitter and target processes are the same) cannot be part of a stable state.

\subsection{Process Hitting translation in ASP}
Before any analysis of the network,
we first need to translate it into ASP\footnote{All programs, including this translation and the methods described in the following, are available online at: \url{https://github.com/EmnaBenAbdallah/verification-of-dynamical-properties_PH}}.
To do this we use the self-describing predicates
\texttt{sort}, \texttt{process} and \texttt{action} to define the sorts, processes and actions of the network, respectively.
\pref{ex:asp-ph} shows how a PH network is defined with these predicates.

\begin{example}[Representation of a PH network in ASP]
\label{ex:asp-ph}
The representation of the PH network of \pref{fig:ph} in ASP is the following:
\begin{lstlisting}
process("a", 0..1). process("b", 0..1). %\label{ASPprocess1}
process("z", 0..1). process("ab", 0..3). %\label{ASPprocess2}
sort(X) :- process(X,_). %\label{ASPsort}
action("a",0,"b",0,1). action("b",1,"b",1,0). %\label{actions1}
action("b",0,"ab",1,0). action("b",0,"ab",3,2). %\label{actions2}
action("b",1,"ab",0,1). action("b",0,"ab",2,3). %\label{actions3}
action("a",0,"ab",2,0). action("a",0,"ab",3,1). %\label{actions4}
action("a",1,"ab",0,2). action("a",1,"ab",1,3). %\label{actions5}
action("ab",3,"z",0,1). %\label{actions6}
\end{lstlisting}
In \refll{ASPprocess1}{ASPprocess2} we create the list of processes corresponding to each sort.
For example the sort ``\texttt{a}'' has 2 processes numbered \texttt{0} and \texttt{1};
predicate ``\texttt{process("a", 0..1).}'' will in fact expand into the two following predicates:
\begin{lstlisting}[numbers=none]
process("a", 0). process("a", 1).
\end{lstlisting}
The processes of the cooperative sort ``\texttt{ab}'',
which represents a cooperation between the biological components ``\texttt{a}'' and ``\texttt{b}'',
have been relabeled \texttt{0}, \texttt{1}, \texttt{2} and \texttt{3}.
\Refl{ASPsort} enumerates every sort of the network from the previous information.
In ASP, a word starting with a capital letter is a variable,
and the underscore (``\texttt{\_}'') is a placeholder for any value.
Finally, all the actions of the network are defined straightforwardly in \refll{actions1}{actions6};
for instance, the predicate ``\texttt{action("a",0,"b",0,1).}'' represents the action
$\PHfrappe{a_0}{b_0}{b_1}$.
\end{example}

\subsection{Search of fixed points}

The enumeration of fixed points requires to translate the definition of a stable state (given in \pref{def:fixpoint})
into an ASP program.
The first step consists of eliminating all processes involved in a self-hit;
the other processes are recorded by the predicate \texttt{shownProcess} (\refll{hiddenProcess}{shownProcess2}).
\begin{lstlisting}
hiddenProcess(A,I) :- action(A,I,A,I,K). %\label{hiddenProcess}
shownProcess(A,I) :- not hiddenProcess(A,I), %\label{shownProcess1}
                     process(A,I). %\label{shownProcess2}
\end{lstlisting}
Then, we have to browse all remaining processes of this graph in order to generate all possible states,
that is, all possible combinations of processes by choosing only one process from each sort (\refll{select-processes1}{select-processes2}).
%So that \refll{hiddenProcess}{shownProcess2} is an optimization to reduce the number of possible states.
\begin{lstlisting}
1 { selectProc(A,I) : shownProcess(A,I) } 1 %\label{select-processes1}
         :- sort(A). %\label{select-processes2}
\end{lstlisting}
The previous lines form a \emph{cardinality rule} that creates as many potential answer sets as the number of possible states
to take into account.
Finally, the last step consists in filtering out any state that is not a fixed point,
or, in other words, eliminating all candidate answer sets in which an action could be played. % (and thus lead the network into another state).
For this, we use a constraint:
any solution that satisfy the body of this constraint will be removed from the answer set.
Regarding our problem, a state is eliminated if there exists an action between two selected processes (\refll{constraintFix1}{constraintFix}).
\begin{lstlisting}
:- action(A,I,B,J,_), selectProc(A,I), %\label{constraintFix1}
         selectProc(B,J), A!=B. %\label{constraintFix}
\end{lstlisting}
In the end, the fixed points of the considered model are exactly the states represented in each remaining answer,
described by the atoms \texttt{selectProc(A,I)}.

\begin{example}[Fixed points enumeration]
The PH model of \pref{fig:ph} contains 4 sorts:
$a$, $b$ and $z$ have 2 processes and $ab$ has 4; therefore, the whole model has $2*2*2*4 = 32$ states (whether they can be reached or not from a given initial state).
We can check that this model contains exactly $2$ fixed points: $\PHstate{b_0, a_1, ab_2, z_0}$ and $\PHstate{b_0, a_1, ab_2, z_1}$.
Indeed, there is no action between each two processes contained in this state so no execution is possible from these.
%In this example, no other state verifies this property.

If we execute the ASP program detailed above (\refll{hiddenProcess}{constraintFix}),
alongside with the description of the PH model given in \pref{ex:asp-ph} (\refll{ASPprocess1}{actions5}),
we obtain two answer sets that match the expected result:
\begin{lstlisting}[numbers=none]
Answer 1 : selectProc(a,1), selectProc(b,0), 
             selectProc(z,0), selectProc(ab,2)
Answer 2 : selectProc(a,1), selectProc(b,0), 
             selectProc(z,1), selectProc(ab,2)
\end{lstlisting}
\end{example}