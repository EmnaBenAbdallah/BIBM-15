\documentclass[conference]{IEEEtran}

% Packages
\usepackage{latexsym}
\usepackage{epic,eepic}
\usepackage{times}
\usepackage{amsmath}
\usepackage{amsfonts}
\usepackage{multirow}
\usepackage[dvips]{graphicx}
\usepackage{algorithm}
\usepackage{algpseudocode}
\usepackage{xcolor}
\usepackage{ulem}
\usepackage{amssymb}

% Tunning
%------------------
\usepackage[12pt]{moresize}
\let\labelindent\relax
\usepackage{enumitem}
\algtext*{EndIf}% Remove "end if" text
\algtext*{EndFor}% Remove "end if" text
%\setlength{\belowcaptionskip}{10.0pt}
%\setlength{\abovecaptionskip}{10.0pt}
%\setlength{\textfloatsep}{10.0pt}
%\setlength{\floatsep}{10.0pt}
% TUNNING
%\setstretch{0.94}
%\fontsize{3.85mm}{3.85mm}\selectfont


\newcommand{\pf}[1]{\langle#1\rangle}
\newcommand{\noproof}{\hfill $qed$}
\newcommand{\ignore}[1]{}

\definecolor{gray50}{gray}{0.15}
\graphicspath{{pictures/}}



\newtheorem{definition}{Definition} % [section]
\newtheorem{example}{Example} % [section]
\newcommand{\pivot}[1]{\mathbin{\, {#1} \,}}
\newcommand{\Pivot}[1]{\mathbin{\; {#1} \;}}
\let\from=\leftarrow

% Pretty ref
\usepackage{prettyref}
\newrefformat{def}{Definition~\ref{#1}}
\newrefformat{fig}{Figure~\ref{#1}}
\newrefformat{tab}{Table~\ref{#1}}
%\newrefformat{pro}{Property~\ref{#1}}
%\newrefformat{pps}{Proposition~\ref{#1}}
%\newrefformat{lem}{Lemma~\ref{#1}}
\newrefformat{th}{Theorem~\ref{#1}}
\newrefformat{sec}{Section~\ref{#1}}
%\newrefformat{subsec}{Subsect.~\ref{#1}}
%\newrefformat{suppl}{Appendix~\ref{#1}}
\newrefformat{ex}{Example~\ref{#1}}
%\newrefformat{eq}{Eq.~\eqref{#1}}
\def\pref{\prettyref}


%Package CAV
\usepackage[english]{babel}
\usepackage{stmaryrd} % Maths (crochets doubles)
\usepackage{url}     % Mise en forme + liens pour URLs
%\usepackage{array}   % Tableaux évolués
\usepackage{setspace}

% Commande perso
\newcommand{\ie}{i.e.,\ }
\newcommand{\eg}{e.g.,\ }
\newcommand{\resp}{resp.\ }


% Citation
\usepackage{cite}

% **** ASP language ***
\usepackage{listings}
\lstdefinelanguage{ASP}{\^^M}
}
% Définition des styles de tous les listings du document
\lstset{
language=ASP,
basicstyle=\small\ttfamily,
columns=fullflexible,
keywordstyle=\bfseries,
firstnumber=last,
keepspaces=true
}
\renewcommand{\thelstnumber}{\the\value{lstnumber}}
%% fin définition

% Styles ASP inline
\newcommand{\ASPnot}{\textbf{not}~}

% les inputs files
% Macros relatives à la traduction de PH avec arcs neutralisants vers PH à k-priorités fixes

% Macros générales
%\newcommand{\ie}{\textit{i.e.} }
\newcommand{\segm}[2]{\llbracket #1; #2 \rrbracket}
%\newcommand{\f}[1]{\mathsf{#1}}

% Notations générales pour PH
\newcommand{\PH}{\mathcal{PH}}
%\newcommand{\PHs}{\mathcal{S}}
\newcommand{\PHs}{\Sigma}
%\newcommand{\PHp}{\mathcal{P}}
\newcommand{\PHp}{\textcolor{red}{\mathcal{P}}}
%\newcommand{\PHproc}{\mathcal{P}}
\newcommand{\PHproc}{\mathbf{Proc}}
\newcommand{\Proc}{\PHproc}
\newcommand{\PHh}{\mathcal{H}}
\newcommand{\PHa}{\PHh}
%\newcommand{\PHa}{\mathcal{A}}
\newcommand{\PHl}{\mathcal{L}}
\newcommand{\PHn}{\mathcal{N}}

\newcommand{\PHhitter}{\mathsf{hitter}}
\newcommand{\PHtarget}{\mathsf{target}}
\newcommand{\PHbounce}{\mathsf{bounce}}
%\newcommand{\PHsort}{\Sigma}
\newcommand{\PHsort}{\PHs}

%\newcommand{\PHfrappeur}{\mathsf{frappeur}}
%\newcommand{\PHcible}{\mathsf{cible}}
%\newcommand{\PHbond}{\mathsf{bond}}
%\newcommand{\PHsorte}{\mathsf{sorte}}
%\newcommand{\PHbloquant}{\mathsf{bloquante}}
%\newcommand{\PHbloque}{\mathsf{bloquee}}

%\newcommand{\PHfrappeR}{\textcolor{red}{\rightarrow}}
%\newcommand{\PHmonte}{\textcolor{red}{\Rsh}}

\newcommand{\PHhitA}{\rightarrow}
\newcommand{\PHhitB}{\Rsh}
%\newcommand{\PHfrappe}[3]{\mbox{$#1\PHhitA#2\PHhitB#3$}}
%\newcommand{\PHfrappebond}[2]{\mbox{$#1\PHhitB#2$}}
\newcommand{\PHhit}[3]{#1\PHhitA#2\PHhitB#3}
\newcommand{\PHfrappe}{\PHhit}
\newcommand{\PHhbounce}[2]{#1\PHhitB#2}
\newcommand{\PHobj}[2]{\mbox{$#1\PHhitB^*\!#2$}}
\newcommand{\PHobjectif}{\PHobj}
\newcommand{\PHconcat}{::}
%\newcommand{\PHneutralise}{\rtimes}
\def\Sce{\mathbf{Sce}}

% Actions plurielles
\newcommand{\PHhitmultsymbol}{\rightarrowtail}
\newcommand{\PHhitmult}[2]{\mbox{$#1 \PHhitmultsymbol #2$}}
\newcommand{\PHfrappemult}{\PHhitmult}
\newcommand{\PHfrappemults}[2]{\PHhitmult{\{#1\}}{\{#2\}}}

\def\PHget#1#2{{#1[#2]}}
%\newcommand{\PHchange}[2]{#1\langle #2 \rangle}
%\newcommand{\PHchange}[2]{(#1 \Lleftarrow #2)}
%\newcommand{\PHarcn}[2]{\mbox{$#1\PHneutralise#2$}}
\newcommand{\PHplay}{\cdot}

\newcommand{\PHstate}[1]{\mbox{$\langle #1 \rangle$}}
\newcommand{\PHetat}{\PHstate}

\def\supp{\mathsf{support}}
\def\first{\mathsf{first}}
\def\last{\mathsf{last}}

\def\DNtrans{\rightarrow_{ADN}}
\def\DNdef{(\mathbb F, \langle f^1, \dots, f^n\rangle)}
\def\DNdep{\mathsf{dep}}
\def\PHPtrans{\rightarrow_{PH}}
\def\get#1#2{#1[{#2}]}
\def\encodeF#1{\mathbf{#1}}
\def\toPH{\encodeF{PH}}
\def\card#1{|#1|}
\def\decode#1{\llbracket#1\rrbracket}
\def\encode#1{\llparenthesis#1\rrparenthesis}
\def\Hits{\PHa}
\def\hit{\PHhit}
\def\play{\cdot}

\def\Pint{\textsc{PINT}}



\usepackage{ifthen}

\newcommand{\currentScope}{}
\newcommand{\currentSort}{}
\newcommand{\currentSortLabel}{}
\newcommand{\currentAlign}{}
\newcommand{\currentSize}{}

\newcounter{la}
\newcommand{\TSetSortLabel}[2]{
  \expandafter\repcommand\expandafter{\csname TUserSort@#1\endcsname}{#2}
}
\newcommand{\TSort}[4]{
  \renewcommand{\currentScope}{#1}
  \renewcommand{\currentSort}{#2}
  \renewcommand{\currentSize}{#3}
  \renewcommand{\currentAlign}{#4}
  \ifcsname TUserSort@\currentSort\endcsname
    \renewcommand{\currentSortLabel}{\csname TUserSort@\currentSort\endcsname}
  \else
    \renewcommand{\currentSortLabel}{\currentSort}
  \fi
  \begin{scope}[shift={\currentScope}]
  \ifthenelse{\equal{\currentAlign}{l}}{
    \filldraw[process box] (-0.5,-0.5) rectangle (0.5,\currentSize-0.5);
    \node[sort] at (-0.2,\currentSize-0.4) {\currentSortLabel};
   }{\ifthenelse{\equal{\currentAlign}{r}}{
     \filldraw[process box] (-0.5,-0.5) rectangle (0.5,\currentSize-0.5);
     \node[sort] at (0.2,\currentSize-0.4) {\currentSortLabel};
   }{
    \filldraw[process box] (-0.5,-0.5) rectangle (\currentSize-0.5,0.5);
    \ifthenelse{\equal{\currentAlign}{t}}{
      \node[sort,anchor=east] at (-0.3,0.2) {\currentSortLabel};
    }{
      \node[sort] at (-0.6,-0.2) {\currentSortLabel};
    }
   }}
  \setcounter{la}{\currentSize}
  \addtocounter{la}{-1}
  \foreach \i in {0,...,\value{la}} {
    \TProc{\i}
  }
  \end{scope}
}

\newcommand{\TTickProc}[2]{ % pos, label
  \ifthenelse{\equal{\currentAlign}{l}}{
    \draw[tick] (-0.6,#1) -- (-0.4,#1);
    \node[tick label, anchor=east] at (-0.55,#1) {#2};
   }{\ifthenelse{\equal{\currentAlign}{r}}{
    \draw[tick] (0.6,#1) -- (0.4,#1);
    \node[tick label, anchor=west] at (0.55,#1) {#2};
   }{
    \ifthenelse{\equal{\currentAlign}{t}}{
      \draw[tick] (#1,0.6) -- (#1,0.4);
      \node[tick label, anchor=south] at (#1,0.55) {#2};
    }{
      \draw[tick] (#1,-0.6) -- (#1,-0.4);
      \node[tick label, anchor=north] at (#1,-0.55) {#2};
    }
   }}
}
\newcommand{\TSetTick}[3]{
  \expandafter\repcommand\expandafter{\csname TUserTick@#1_#2\endcsname}{#3}
}

\newcommand{\myProc}[3]{
  \ifcsname TUserTick@\currentSort_#1\endcsname
    \TTickProc{#1}{\csname TUserTick@\currentSort_#1\endcsname}
  \else
    \TTickProc{#1}{#1}
  \fi
  \ifthenelse{\equal{\currentAlign}{l}\or\equal{\currentAlign}{r}}{
    \node[#2] (\currentSort_#1) at (0,#1) {#3};
  }{
    \node[#2] (\currentSort_#1) at (#1,0) {#3};
  }
}
\newcommand{\TSetProcStyle}[2]{
  \expandafter\repcommand\expandafter{\csname TUserProcStyle@#1\endcsname}{#2}
}
\newcommand{\TProc}[1]{
  \ifcsname TUserProcStyle@\currentSort_#1\endcsname
    \myProc{#1}{\csname TUserProcStyle@\currentSort_#1\endcsname}{}
  \else
    \myProc{#1}{process}{}
  \fi
}

\newcommand{\repcommand}[2]{
  \providecommand{#1}{#2}
  \renewcommand{#1}{#2}
}
\newcommand{\THit}[5]{
  \path[hit] (#1) edge[#2] (#3#4);
  \expandafter\repcommand\expandafter{\csname TBounce@#3@#5\endcsname}{#4}
}
\newcommand{\TBounce}[4]{
  (#1\csname TBounce@#1@#3\endcsname) edge[#2] (#3#4)
}

%\newcommand{\TState}[1]{
%  \foreach \proc in {#1} {
%    \node[current process] (\proc) at (\proc.center) {};
%  }
%}

\newcommand{\TState}[1]{
  \foreach \proc in {#1} {
        \node[current process] (\proc) at (\proc.center) {};
  };
}
\newcommand{\TCoopHit}[6]{
  \node[#2, apdot] at (#3) {};
  \foreach \proc in {#1} {
    \draw[#2,-] (#3) edge (\proc);
  }
  \path[hit] (#3) edge[#2] (#4#5);
  \expandafter\repcommand\expandafter{\csname TBounce@#4@#6\endcsname}{#5}
}

% ex : \TAction{c_1}{a_1.west}{a_0.north west}{}{right}
% #1 = frappeur
% #2 = cible
% #3 = bond
% #4 = style frappe
% #5 = style bond
\newcommand{\TAction}[5]{
  \THit{#1}{#4}{#2}{}{#3}
  \path[bounce, bend #5=50] \TBounce{#2}{}{#3}{};
}

% ex : \TActionPlur{f_1, c_0}{a_0.west}{a_1.south west}{}{3.5,2.5}{left}
% #1 = frappeur
% #2 = cible
% #3 = bond
% #4 = style frappe
% #5 = coordonnées point central
% #6 = direction bond
\newcommand{\TActionPlur}[6]{
  \TCoopHit{#1}{#4}{#5}{#2}{}{#3}
  \path[bounce, bend #6=50] \TBounce{#2}{}{#3}{};
}

% Styles TikZ et couleurs personnalisées

\usepackage{tikz}

\newdimen\pgfex
\newdimen\pgfem
\usetikzlibrary{arrows,shapes,shadows,scopes}
\usetikzlibrary{positioning}
\usetikzlibrary{matrix}
\usetikzlibrary{decorations.text}
\usetikzlibrary{decorations.pathmorphing}
\usetikzlibrary{arrows,shapes}

\definecolor{lightgray}{rgb}{0.8,0.8,0.8}
\definecolor{lightgrey}{rgb}{0.8,0.8,0.8}

\definecolor{lightred}{rgb}{1,0.8,0.8}
\definecolor{lightgreen}{rgb}{0.7,1,0.7}
\definecolor{darkgreen}{rgb}{0,0.5,0}
\definecolor{darkblue}{rgb}{0,0,0.5}
\definecolor{darkyellow}{rgb}{0.5,0.5,0}
\definecolor{lightyellow}{rgb}{1,1,0.6}
\definecolor{darkcyan}{rgb}{0,0.6,0.6}
\definecolor{lightcyan}{rgb}{0.6,1,1}
\definecolor{darkorange}{rgb}{0.8,0.2,0}
\definecolor{notsodarkred}{rgb}{0.8,0,0}

\definecolor{notsodarkgreen}{rgb}{0,0.7,0}

%\definecolor{coloract}{rgb}{0,1,0}
%\definecolor{colorinh}{rgb}{1,0,0}
\colorlet{coloract}{darkgreen}
\colorlet{colorinh}{red}
\colorlet{coloractgray}{lightgreen}
\colorlet{colorinhgray}{lightred}
\colorlet{colorinf}{darkgray}
\colorlet{coloractgray}{lightgreen}
\colorlet{colorinhgray}{lightred}

\colorlet{colorgray}{lightgray}
\colorlet{colorhl}{blue}


\tikzstyle{boxed ph}=[]
\tikzstyle{sort}=[fill=lightgray, rounded corners, draw=black]
\tikzstyle{process}=[circle,draw,minimum size=15pt,fill=white,font=\footnotesize,inner sep=1pt]
%\tikzstyle{black process}=[process, draw=blue, fill=red,text=black,font=\bfseries]
\tikzstyle{gray process}=[process, draw=black, fill=lightgray]
\tikzstyle{highlighted process}=[current process, fill=gray]
\tikzstyle{process box}=[fill=none,draw=black,rounded corners]
%\tikzstyle{current process}=[process, draw=black, fill=lightgray]
\tikzstyle{current process}=[process,fill=lightgray]
\tikzstyle{hl process}=[process,fill=blue!30]
\tikzstyle{tick label}=[font=\footnotesize]
\tikzstyle{tick}=[densely dotted] %-
\tikzstyle{hit}=[->,>=angle 45]
\tikzstyle{selfhit}=[min distance=50pt,curve to]
\tikzstyle{bounce}=[densely dotted,>=stealth',->]
\tikzstyle{ulhit}=[draw=lightgray,fill=lightgray]
\tikzstyle{pulhit}=[fill=lightgray]
\tikzstyle{bulhit}=[draw=lightgray]
\tikzstyle{hl}=[very thick,colorhl]
\tikzstyle{hlb}=[very thick]
\tikzstyle{hlhit}=[hl]
%\tikzstyle{hl2}=[hl]
%\tikzstyle{nohl}=[font=\normalfont,thin]

\tikzstyle{update}=[draw,->,dashed,shorten >=.7cm,shorten <=.7cm]

\tikzstyle{unprio}=[draw,thin]%[double]
%\tikzstyle{prio}=[draw,thick,-stealth]%[double]
\tikzstyle{prio}=[draw,-stealth,double]

\tikzstyle{hitless graph}=[every edge/.style={draw=red,-}]

\tikzstyle{aS}=[every edge/.style={draw,->,>=stealth}]
\tikzstyle{Asol}=[draw,circle,minimum size=5pt,inner sep=0,node distance=1cm]
\tikzstyle{Aproc}=[draw,node distance=1.2cm]
\tikzstyle{Aobj}=[node distance=1.5cm]
\tikzstyle{Anos}=[font=\Large]

\tikzstyle{AsolPrio}=[Asol,double]
\tikzstyle{AprocPrio}=[Aproc,double]
\tikzstyle{aSPrio}=[aS,double]

\colorlet{colorhlwarn}{notsodarkred}
\colorlet{colorhlwarnbg}{lightred}
\tikzstyle{Ahl}=[very thick,fill=colorhlwarnbg,draw=colorhlwarn,text=colorhlwarn]
\tikzstyle{Ahledge}=[very thick,double=colorhlwarnbg,draw=colorhlwarn,color=colorhlwarn]





%\definecolor{darkred}{rgb}{0.5,0,0}



\tikzstyle{grn}=[every node/.style={circle,draw=black,outer sep=2pt,minimum
                size=15pt,text=black}, node distance=1.5cm, ->]
\tikzstyle{inh}=[>=|,-|,draw=colorinh,thick, text=black,label]
\tikzstyle{act}=[->,>=triangle 60,draw=coloract,thick,color=coloract]
\tikzstyle{inhgray}=[>=|,-|,draw=colorinhgray,thick, text=black,label]
\tikzstyle{actgray}=[->,>=triangle 60,draw=coloractgray,thick,color=coloractgray]
\tikzstyle{inf}=[->,draw=colorinf,thick,color=colorinf]
%\tikzstyle{elabel}=[fill=none, above=-1pt, sloped,text=black, minimum size=10pt, outer sep=0, font=\scriptsize,draw=none]
\tikzstyle{elabel}=[fill=none,text=black, above=-2pt,%sloped,
minimum size=10pt, outer sep=0, font=\scriptsize, draw=none]
%\tikzstyle{elabel}=[]


\tikzstyle{plot}=[every path/.style={-}]
\tikzstyle{axe}=[black,->,>=stealth']
\tikzstyle{ticks}=[font=\scriptsize,every node/.style={black}]
\tikzstyle{mean}=[thick]
\tikzstyle{interval}=[line width=5pt,red,draw opacity=0.7]
%\definecolor{lightred}{rgb}{1,0.3,0.3}

%\tikzstyle{hl}=[yellow]
%\tikzstyle{hl2}=[orange]

%\tikzstyle{every matrix}=[ampersand replacement=\&]
%\tikzstyle{shorthandoff}=[]
%\tikzstyle{shorthandon}=[]
\tikzstyle{objective}=[process,very thick,fill=yellow!50]

\tikzstyle{coopupdate}=[-stealth,decorate,decoration={zigzag,amplitude=1.5pt,post=lineto,post length=.3cm,pre=lineto,pre length=.3cm}]

\tikzstyle{labelprio}=[circle, fill=blue!30, inner sep=0pt, minimum size=13pt]
\tikzstyle{labelprio1}=[labelprio]
\tikzstyle{labelprio2}=[labelprio, fill=red!60]
\tikzstyle{labelprio3}=[labelprio, fill=orange!50]
\tikzstyle{labelprio4}=[labelprio, fill=brown!50]

\tikzstyle{labelstocha}=[rectangle, rounded corners=4pt]

\tikzstyle{andot}=[circle, fill=black, inner sep=1.2pt, draw=transparent]
\tikzstyle{anligne}=[thick]

\tikzstyle{apdot}=[andot] %[circle, fill=black, draw=black, inner sep=1]
\tikzstyle{apdotsimple}=[] %[circle, fill=black, draw=black, inner sep=1]

% Figure de résumé des liens entre les formalismes
\tikzstyle{equiv-externe}=[thick, rounded corners, draw=gray, fill=gray!10, align=center,
  inner sep=8]


% correct bad hyphenation here
\hyphenation{op-tical net-works semi-conduc-tor}


\begin{document}
%
% paper title
% can use linebreaks \\ within to get better formatting as desired
\title{Exhaustive analysis of dynamical properties through simulation of biological regulatory networks}


% author names and affiliations
% use a multiple column layout for up to three different
% affiliations

% conference papers do not typically use \thanks and this command
% is locked out in conference mode. If really needed, such as for
% the acknowledgment of grants, issue a \IEEEoverridecommandlockouts
% after \documentclass

% for over three affiliations, or if they all won't fit within the width
% of the page, use this alternative format:
% 
\author{\IEEEauthorblockN{Emna Ben Abdallah\IEEEauthorrefmark{1}, Maxime Folschette\IEEEauthorrefmark{2}, Olivier Roux\IEEEauthorrefmark{1} and Morgan Magnin\IEEEauthorrefmark{3} }

\IEEEauthorblockA{\IEEEauthorrefmark{1}LUNAM Universit\'e, \'Ecole Centrale de Nantes, IRCCyN UMR CNRS 6597\\
(Institut de Recherche en Communications et Cybern\'etique de Nantes), 1 rue de la No\"e, 44321 Nantes, France.\\ 
 Email: emna.ben-abdallah@irccyn.ec-nantes.fr \\
 olivier.roux@irccyn.ec-nantes.fr
 }
\IEEEauthorblockA{\IEEEauthorrefmark{2}School of Electrical Engineering and Computer Science, University of Kassel, Germany \\
Email: maxime.folschette@uni-kassel.de}
\IEEEauthorblockA{\IEEEauthorrefmark{3}National Institute of Informatics, 2-1-2, Hitotsubashi, Chiyoda-ku, Tokyo 101-8430, Japan.\\
 Email: morgan.magnin@irccyn.ec-nantes.fr
 }}




% use for special paper notices
%\IEEEspecialpapernotice{(Invited Paper)}




% make the title area
\maketitle


\begin{abstract}
The regulation of gene expression is achieved through genetic regulatory networks structured by systems of interactions between cells, DNA, RNA and proteins. As most genetic regulatory networks of interest involve many components connected through positive and negative reactions, an intuitive understanding of their dynamics is hard to obtain. As a consequence, formal methods and computer tools for the modeling and simulation of genetic regulatory networks are indispensable. The complexity of Biological Regulatory Networks (BRNs) often defies the intuition of the biologist and calls for the development of proper mathematical methods to model their structures and to delineate their dynamical properties.  Our new formalism the Process Hitting (PH) defined as a restriction of synchronous automata networks is notably suitable. Also it is not limited, to model and analyze efficiently BRNs.
In this paper, we propose a new logical approach to perform model-checking of dynamical properties of biological regulatory networks modeled in PH. 
Our work here focuses on state reachability properties on the one hand and on the identification of fixed points on the other hand. The originality of our model-checking approach relies in the exhaustive enumeration of all possible solutions of a simulation as well as of a dynamical property.
The merits of our methods is illustrated by applying them to biological examples of various sizes and comparing the results with some existing approaches.
It turns out that our approach succeeds in processing large models with a high number of components and interactions.

\end{abstract}

\IEEEpeerreviewmaketitle



\section{Introduction}
The paper is motivated by the problem of steady states identification as well as reachability problem in biological regulatory systems that describes genes and proteins interactions. Indeed the regulatory phenomena play a crucial role in biological systems, they need to be studied accurately. That’s why the study of Biological Regulatory Networks (BRNs) has been the subject of numerous researches. BRNs consist in sets of either positive or negative mutual effects between the components. With the purpose of analyzing these systems, they are often modeled as graphs which makes it possible to determine the possible evolutions of all the interacting components of the system. Thus, in order to address the formal checking of dynamical properties within very large BRNs, new formalisms and conversely new techniques have been proposed during the last decade. For example, Boolean networks \cite{stuart1993origins} \cite{kauffman1969metabolic} have been used due to their simplicity and ability to handle noisy data but lose data information by having a binary representation of the genes. Besides, artificial neural networks omit using a hidden layer so that they can be interpreted, losing the ability to model higher order correlations in the data. We can cite some works that used these models and which have been developed to verify the dynamic properties of BRNs modeled : GINsim (Gene Interaction Network simulation) \cite{chaouiya2012logical, gonzalez2006ginsim} is a computer tool for the modeling and simulation of genetic regulatory networks. It allows the user to simulate a model and/or analyse its qualitative dynamical behavior. The inconvenient of this tool is the difficulty of analyzing the big networks. Indeed, in order to verifying a dynamical property it is necessary to compute the boolean network from the corresponding Thomas network. We can cite also libDDD (Library of Data Decision Diagrams) a library for symbolic model-checking of CTL \& LTL properties. It can thus especially be used to check reachability properties. But it cannot output the execution path solving the reachability.The computing tests show that \textsc{LibDDD} and \textsc{GINsim} take a long time to respond, and gets out of memory for the big examples (more than 40 components).  %à revoir
%Several other promising modeling and analysing techniques have been used, including Boolean networks, Petri nets, Bayesian networks, we can cite {ref des autres travaux sur l’atteignabilité} \ref{paoletti2014analyzing}

It is also common to model biological networks with a set of coupled ordinary differential equations (ODEs) \cite{chu2009models}, describing the kinetic reactions. But the temporal evolution of the systems modeled by ODEs is computed by a complex derivation approach. This complexity of computing the evolution so that the verification of the dynamic properties causes also a long time to respond. 

New experimental results \cite{elowitz2002stochastic} \cite{blake2003noise} demonstrated that gene expression is a stochastic process. Thus, many biologists and bioinformaticians are now using the stochastic formalism \cite{arkin1998stochastic}. This formalism permits to represent each gene expression \cite{raser2005noise} and small synthetic genetic networks, \cite{elowitz2000synthetic} \cite{gardner2000construction}. Such that our new formalism named Process Hitting (PH) \cite{PMR10-TCSB}. It models concurrent systems having components with a few qualitative levels. Furthermore, PH is suitable, according to the precision of the available information, to model BRNs with different levels of abstraction by capturing the most general dynamics.
Thus, in order to address the formal checking of dynamical properties within very large BRNs, we recently introduced a new formalism, named the "Process Hitting'' (PH) \cite{PMR10-TCSB}, to model concurrent systems having components with a few qualitative levels. A PH describes, in an atomic manner, the possible evolutions of a "process'' (representing one component at one level) triggered by the hit of other "processes’’ in the system. Comparing with other models of BRNs, this particular structure of PH makes the formal analysis of BRNs with hundreds of components easier to be tractable. This was proved by a first work on the PH in \cite{PMR12-MSCS} which analysis big networks and gives a response through a short time. But this developed technic was based on abstract methods computing approximations of the dynamics that can be inconclusive in some cases. Moreover, in the case of a positive answer, it currently does not return the execution of the path achieving the desired reachability, but only outputs its conclusion.

Our goal in this paper is to develop exhaustive methods that analyze Biological Regulatory Network modeled in Process Hitting. We will prove that these methods are exhaustive and fast. In addition it returns the execution path for the reachability problem and all the stable states of a model.

The paper is organised as follows % à remplir
We conclude the paper with a short discussion. 

\section{Preliminary definitions}
\label{sec:defs}
\subsection{Process Hitting}
\label{sec:prem-def}

\pref{def:PH} introduces the Process Hitting~(PH)~\cite{PMR10-TCSB}
which allows to model a finite number of local levels,
called \emph{processes},
grouped into a finite set of components, called \emph{sorts}.
A process is noted $a_i$, where $a$ is the sort's name,
and $i$ is the process identifier within sort $a$.
At any time, exactly one process of each sort is \emph{active},
and the set of active processes is called a \emph{state}.

The concurrent interactions between processes are defined by a set of \emph{actions}.
Each action is responsible for the replacement of one process by another of the same sort
conditioned by the presence of at most one other process in the current state.
An action is denoted by $\PHfrappe{a_i}{b_j}{b_k}$, which is read as
``$a_i$ \emph{hits} $b_j$ to make it \emph{bounce} to $b_k$'',
where $a_i$, $b_j$, $b_k$ are processes of sorts $a$ and $b$,
called respectively \emph{hitter}, \emph{target} and
\emph{bounce} of the action.
We also call a \emph{self-hit} any action whose hitter and target sorts are the same,
that is, of the form: $\PHfrappe{a_i}{a_i}{a_k}$.

The PH is therefore a restriction of synchronous automata, where each transition
changes the local state of exactly one automaton,
and is triggered by the local states of at most two distinct automata.
This restriction in the form of the actions was chosen to permit
the development of efficient static analysis methods
based on abstract interpretation~\cite{PMR12-MSCS} as well as the dynamic analysis that we present in this paper. In addition due to this restriction we need to add non biological components into the system that allow to present the cooperation between $2$ or more components which we call \emph{cooperative sorts}. So it maintains the effectiveness of the static and dynamic analysis.

\begin{definition}[Process Hitting]\label{def:PH}
  A \emph{Process Hitting} is a triple $(\PHs,\PHl,\PHa)$ where:
  \begin{itemize}
    \item  $\PHs = \{a,b,\dots\}$ is the finite set of \emph{sorts};
    \item  $\PHl = \prod_{a\in\PHs} \PHl_a$ is the set of \emph{states} where
      $\PHl_a = \{a_0,\dots,a_{l_a}\}$
      is the finite set of \emph{processes} of sort $a\in\Sigma$
      and $l_a$ is a positive integer, with $a\neq b\Rightarrow \PHl_a \cap \PHl_b = \emptyset$;
    \item  $\PHa = \{ \PHfrappe{a_i}{b_j}{b_k} \in \PHl_a \times \PHl_b^2 \mid
      (a,b) \in \PHs^2 \wedge b_j\neq b_k \wedge a=b\Rightarrow a_i=b_j\}$
      is the finite set of \emph{actions}.
  \end{itemize}
\end{definition}

\begin{example}

\tikzstyle{prio}=[draw,thick,-stealth]

\begin{figure}[ht]
\pref{fig:ph} represents a PH $(\PHs,\PHl,\PHa)$ with four sorts
  \centering
  \scalebox{1.4}{
  \begin{tikzpicture}
    \TSort{(0,0)}{a}{2}{b}
    \TSort{(0,4)}{b}{2}{t}
    \TSort{(5,0.5)}{z}{2}{r}

    \TSetTick{ab}{0}{00}
    \TSetTick{ab}{1}{01}
    \TSetTick{ab}{2}{10}
    \TSetTick{ab}{3}{11}
    \TSort{(0.5,2)}{ab}{4}{t}

    \THit{a_0}{}{ab_3}{.south west}{ab_1}
    \THit{a_0}{}{ab_2}{.south west}{ab_0}
    \THit{a_1}{}{ab_1}{.south}{ab_3}
    \THit{a_1}{}{ab_0}{.south}{ab_2}

    \THit{b_0}{}{ab_3}{.north}{ab_2}
    \THit{b_0}{}{ab_1}{.north}{ab_0}
    \THit{b_1}{}{ab_2}{.north}{ab_3}
    \THit{b_1}{}{ab_0}{.north}{ab_1}
    
    \THit{b_1}{selfhit}{b_1}{.north west}{b_0}
    \THit{a_0}{bend left}{b_0}{.south west}{b_1}

    \THit{ab_3}{}{z_0}{.west}{z_1}

    \path[bounce, bend right]
     \TBounce{ab_1}{}{ab_3}{.south}
     \TBounce{ab_0}{}{ab_2}{.south}
     \TBounce{ab_3}{}{ab_2}{.east}
     \TBounce{ab_1}{}{ab_0}{.east}
    ;
    \path[bounce, bend left]
      \TBounce{ab_3}{}{ab_1}{.south east}
      \TBounce{ab_2}{}{ab_0}{.south east}
      \TBounce{ab_2}{}{ab_3}{.west}
      \TBounce{ab_0}{}{ab_1}{.west}
    ;
    \path[bounce, bend right]
      \TBounce{b_1}{}{b_0}{.north east}
      \TBounce{b_0}{}{b_1}{.south west}
    ;
    \path[bounce, bend left]
      \TBounce{z_0}{}{z_1}{.south west}
    ;
    \TState{a_0, b_0, ab_0, z_0}
  \end{tikzpicture}
  }
  \caption{\label{fig:ph}
A PH model example with four sorts: $a$, $b$, $ab$ and $z$. Boxes represent the sorts (network components), circles represent the processes (component levels), and the actions that model the dynamic behavior are depicted by pairs of arrows in solid and dotted lines. $a$ is either at level 0 or 1, $b$ at either level 0 or 1, $z$ at either level 0, 1 or 2 and the cooperative sort $ab$ has 4 levels corresponding to the combinaison of the levels of the corresponding sorts $a$ and $b$. The grayed processes stand for the possible initial state $\PHstate{a_0, b_0, ab_{00}, z_0}$.
  }
\end{figure}


\end{example}

A state of the networks is a set of active processes containing a single process of each sort.
The active process of a given sort $a \in \PHs$ in a state $s \in \PHl$
is noted $\PHget{s}{a}$.
For any given process $a_i$ we also note: $a_i \in s$ if and only if $\PHget{s}{a} = a_i$. It means that the component, or the \emph{sort}, $a$ is at level $i$ during the state $s$.

\begin{definition} [Playable action]
\label{def:playableAction}
Let $\PH = (\PHs,\PHl,\PHa)$ be a Process Hitting and $s \in \PHl$ a state of $PH$.
We say that the action $h = \PHfrappe{a_i}{b_j}{b_k} \in \PHa$
is \emph{playable in state $s$} if and only if
$a_i \in s$ and $b_j \in s$ (\ie $\PHget{s}{a} = a_i$ and $\PHget{s}{b}=b_j$).
The resulting state after playing $h$ in $s$
is called a \emph{successor} of $s$ and
is denoted by $(s \play h)$,
where $\PHget{(s \play h)}{b} = b_k$ and
$\forall c \in \PHs, c \neq b \Rightarrow \PHget{(s \play h)}{c}=\PHget{s}{c}$.
\end{definition}

PH was chosen for several reasons. It is a general framework that was mainly used for biological networks. Besides, it allows to represent any kind of dynamical model,
and converters to several other representations are available and included into \textsc{Pint}\footnote{\textsc{Pint} version 2015-11-14: \url{http://loicpauleve.name/pint/}}~\cite{PMR12-MSCS}. Espacially for the particular form of the actions in a PH model allows
to easily represent them using the Answer Set Programming \cite{Baral03, Vladimir, Glimpse, sureshkumar2006ansprolog},
with one fact per action. We note that although an efficient dynamical analysis already exists for this framework,
based on an approximation of the dynamics,
it is interesting to identify its limits
and compare them to the exhaustive approach we present in this paper.

\subsection{Dynamical properties}

The study of the dynamics of biological networks was the focus of many works, explaining the diversity of network modelings and the different methods developed in order to check dynamic properties.
In this paper we focus on 2 main properties: the stable states and the reachability.
In the following, we consider a PH model $(\PHs,\PHl,\PHa)$,
and we formally define these properties
and explain how they could be verified on such a network.

The notion of \emph{fixed point}, also called \emph{stable state},
is given in \pref{def:fixpoint}.
A fixed point is a state which has no successor.
Such states have a particular interest as they denote states in which the model
stays indefinitely,
and the existence of several of these states denotes a switch in the dynamics~\cite{wuensche1998genomic}.

\begin{definition}[Fixed point]
\label{def:fixpoint}
  A state $s \in \PHl$ is called a \emph{fixed point}
  (or equivalently \emph{stable state})
  if and only if it has no successors.
  In other words, and in $PH$ symantic, $s$ is a fixed point if and only if no action is playable in this state:
  \[\forall \PHfrappe{a_i}{b_j}{b_k} \in \PHa, a_i \notin s \vee b_j \notin s \enspace.\]
\end{definition}

A finer and more interesting dynamical property consists in
the notion of \emph{reachability}.
Such a property, defined in \pref{def:reachability},
states that starting from a given initial state, it is possible
to reach a given goal, that is, a state that contains a process
or a set of processes.
Checking such a dynamical property is considered difficult
as in usual model-checking techniques,
it is required to build (a part of) the state graph,
which has an exponential complexity.

In the following, if $s \in \PHl$ is a state,
we call \emph{scenario in $s$}
any sequence of actions that are successively playable in $s$.
We also note $\Sce(s)$ the set of all scenarios in $s$.
Moreover, we denote by $\Proc = \bigcup_{a \in \PHs} \PHl_a$
the set of all process in $\PH$.

\begin{definition}[Reachability property]
\label{def:reachability}
  If $s \in \PHl$ is a state and $A \subseteq \Proc$ is a set of processes,
  we denote by $\mathcal{P}(s, A)$ the following \emph{reachability property}:
  \[\mathcal{P}(s, A) \equiv \exists \delta \in \Sce(s), \forall a_i \in A, \PHget{(s \play \delta)}{a} = a_i
    \enspace.\]
\end{definition}

The rest of the paper focuses on the resolution of the previous issues. We demonstrate our algorithms about: the simulation of a biological regulatory network modeled in PH,
the enumeration of all fixed points (\pref{sec:fixpoint} and the verification of the reachability property will be handled in XXXXX (\pref{sec:dynamics}).


\section{Fixed point enumeration}
\label{sec:fixpoint}
The study of fixed points (and, more generally, basins of attraction) provides an important understanding of the different behaviors of a Biological Regulatory Network (BRN)~\cite{wuensche1998genomic}.
Indeed, a system will always eventually end in a basin of attraction,
and this may depend on biological switch or other complex phenomena.
A fixed point is a state of the BRN in which it is not possible to have any new dynamic evolutions anymore;
in other words, it is a basin of attraction that is composed of only one state.

In the following, we consider a Process Hitting $\PH = (\Sigma, \PHl, \PHh)$.
A state $s \in L$ is a fixed point (or stable state) of the Process Hitting model if and only if it has no successor state, \ie regarding the active processes in $s$ there is no playable action (see \pref{def:fixpoint}).
Therefore, it has been shown~\cite{PMR10-TCSB} that
a stable state in a Process Hitting network is a state so that
every active process does not hit or is not hit by another process in the same state.
We note especially that given this result, processes involved in a self-hit (an action whose hitter and target processes are the same) cannot be part of a stable state.

\subsection{Process Hitting translation in ASP}
Before analyzing the dynamics of the network,
we first need to translate the concerned PH network into ASP\footnote{All programs, including this translation and the methods described in the following, are available online at: \url{https://github.com/EmnaBenAbdallah/verification-of-dynamical-properties_PH}}.
To do this we use the following self-describing predicates:
\texttt{sort} to define sorts, \texttt{process} for the processes and \texttt{action} for the network actions.
\pref{ex:asp-ph} shows how a PH network is defined with these predicates.

\begin{example}[Representation of a PH network in ASP]
\label{ex:asp-ph}
The representation of the PH network of \pref{fig:ph} in ASP is the following:
\begin{lstlisting}
process("a", 0..1). process("b", 0..1). %\label{ASPprocess1}
process("z", 0..1). process("ab", 0..3). %\label{ASPprocess2}
sort(X) :- process(X,_). %\label{ASPsort}
action("a",0,"b",0,1). action("b",1,"b",1,0). %\label{actions1}
action("b",0,"ab",1,0). action("b",0,"ab",3,2). %\label{actions2}
action("b",1,"ab",0,1). action("b",0,"ab",2,3). %\label{actions3}
action("a",0,"ab",2,0). action("a",0,"ab",3,1). %\label{actions4}
action("a",1,"ab",0,2). action("a",1,"ab",1,3). %\label{actions5}
action("ab",3,"z",0,1). %\label{actions6}
\end{lstlisting}
In \refll{ASPprocess1}{ASPprocess2} we create the list of processes (expression level) corresponding to each sort,
for example the sort ``\texttt{a}'' has 2 processes numbered \texttt{0} and \texttt{1};
this predicate will in fact expand into the two following predicates:
\begin{lstlisting}[numbers=none]
process("a", 0). process("a", 1).
\end{lstlisting}
The processes of the cooperative sort ``\texttt{ab}'',
which represents a cooperation between the biological components ``\texttt{a}'' and ``\texttt{b}'',
have been relabeled \texttt{0}, \texttt{1}, \texttt{2} and \texttt{3}.
\Refl{ASPsort} enumerates every sort of the network from the previous information.
In ASP, a word starting with a capital letter is a variable,
and the underscore (``\texttt{\_}'') is a placeholder for any value.
Finally, all the actions of the network are defined straightforwardly in \refll{actions1}{actions6};
for instance, the predicate \texttt{action("a",0,"b",0,1)} represents the action
$\PHfrappe{a_0}{b_0}{b_1}$.
\end{example}

\subsection{Search of fixed points}

The enumeration of fixed points requires to translate the definition of a stable state (see \pref{def:fixpoint})
into an ASP program.
The first step consists of eliminating all processes involved in a self-hit;
the other processes are recorded by the predicate \texttt{shownProcess} (\refll{hiddenProcess}{shownProcess2}).
\begin{lstlisting}
hiddenProcess(A,I) :- action(A,I,B,J,K), A=B. %\label{hiddenProcess}
shownProcess(A,I) :- not hiddenProcess(A,I), %\label{shownProcess1}
                     process(A,I). %\label{shownProcess2}
\end{lstlisting}
Then, we have to browse all remaining processes of this graph in order to generate all possible states,
that is, all possible combinations of processes by choosing only one process from each sort (\refll{select-processes1}{select-processes2}).
%So that \refll{hiddenProcess}{shownProcess2} is an optimization to reduce the number of possible states.
\begin{lstlisting}
1 { selectedProcess(A,I) : shownProcess(A,I) } 1 %\label{select-processes1}
         :- sort(A). %\label{select-processes2}
\end{lstlisting}
The previous cardinality rule creates as many potential answer sets as there are possible states
to take into account.
Finally, the last step consists in filtering out any state that is not a fixed point,
or, in other words, eliminating all candidate answer sets in which an action could be played (and thus lead the network into another state).
For this, we use a constraint:
any solution that satisfy the body of this constraint will be removed from the answer set.
Regarding our problem, a state is eliminated if there exists an action between two selected processes (\refll{constraintFix1}{constraintFix}).
\begin{lstlisting}
:- action(A,I,B,J,_), selectedProcess(A,I), %\label{constraintFix1}
    selectedProcess(B,J), A!=B. %\label{constraintFix}
\end{lstlisting}
In the end, the fixed points of the considered model are exactly the states represented in each remaining answer,
described by the atoms \texttt{fixProc(A,I)} (\refl{fixproc}).
% Finally the \texttt{selectedProcess(A,I)} permits to define the fixed point which is compound by these selected processes:
\begin{lstlisting}
fixProc(A,I) :- selectedProcess(A,I). %\label{fixproc}
\end{lstlisting}

\begin{example}[Fixed points enumeration]
The PH model of \pref{fig:ph} contains 4 sorts:
$a$, $b$ and $z$ have 2 processes and $ab$ has 4; therefore, the whole model has $2*2*2*4 = 32$ states (whether they can be reached or not from a given initial state).
We can check that this model contains $2$ fixed points: $\PHstate{b_0, a_1, ab_2, z_0}$ and $\PHstate{b_0, a_1, ab_2, z_1}$.
Indeed, there is no action between each two processes contained in this state so no execution is possible from these.
In this example, no other states verify this property.

If we execute the ASP program detailed below (see \refll{hiddenProcess}{constraintFix}),
alongside with the description of the PH model given in \pref{ex:asp-ph} (see \refll{ASPprocess1}{actions5}),
we obtain two answer sets that match the expected result:
\begin{lstlisting}[numbers=none]
Answer 1 : fixProc(a,1), fixProc(b,0), 
		fixProc(z,0), fixProc(ab,2)
Answer 2 : fixProc(a,1), fixProc(b,0), 
		fixProc(z,1), fixProc(ab,2)
\end{lstlisting}
\end{example}

\section{Dynamical analysis}
\label{sec:dynamics}
%Section: Dynamic network evolution

In this section, we present at first how to determine the possible behaviour in a PH model after a finite number of steps with an ASP program.
Then we tackle the reachability question: are there scenarios from a given initial state
that allow to reach a given goal? If yes is it possible to display the shortest path?

\subsection{Network simulation}
In the previous section, enumerating the fixed points did not require to
encode the full dynamics of PH, but only a condition. It is a static verification.
In this section, we thus implement a dynamic simulation of the PH into ASP. Then it permits to apply an exhaustive analysis to search for the paths allowing to reach the goals.

Firstly, we focus on the simulation, evolution of models in a limited number of steps.
We therefore define the predicate \texttt{time(0..n)} which sets the number of steps we want to play.
The value of \texttt{n} can be arbitrarily chosen;
for example, \texttt{time(0..10)} will compute the 11 first steps,
including the initial state.
In order to specify such an initial state, we add several facts
to make a list of all processes included in this state:
\begin{lstlisting}
init(activeProcess("a",0)). 
init(activeProcess("b",0)).
init(activeProcess("ab",0)).
init(activeProcess("z",0)).
\end{lstlisting}
where \texttt{"a"} is the name of the sort and \texttt{"0"} the index of the active process.
The dynamics of a network is described by its actions;
therefore, identifying the future states requires to first identify the playable actions for each state.
We remind that an action is playable in a state when both its hitter process and target process are active in this state (see \pref{def:playableAction}).
Therefore, we define an ASP predicate \texttt{playable(action(A,I,B,J,K),T)} that is true
when the processes $\texttt{A}_\texttt{I}$ and $\texttt{B}_\texttt{J}$ are active at step \texttt{T}.

The cardinality rule of lines \ref{e2a}
creates a set of as many predicates as there are possible evolutions from the current step,
thus reproducing all possible branchings in the possible evolutions of the model in the form of as many potential answer sets. It is also needed to enforce the strictly asynchronous dynamic
which state that exactly one process can change between two steps.
We thus represent the change of the active process of a sort
by the predicate \texttt{change(B,T+1)}
which means that in sort \texttt{B}, the active process can change between steps \texttt{T} and \texttt{T+1}.
To remove all scenarios where two or more actions have been played between
two steps, we use the constraint of line \ref{e2}.
Thus, the remaining scenarios contained in the answer sets all strictly follow
the asynchronous dynamics of the PH.

\begin{lstlisting}
0{play(Action,T)}1 :- playable(Action,T), time(T). %\label{e2a} 
:- 2{play(Action,T)}, time(T). %\label{e2}
change(B,T+1) :- play(action(_,_,B,_,_),T),time(T). %\label{e3}
\end{lstlisting}

Finally, the active processes at step \texttt{T+1},
that represent the next network state depending on the chosen bounce,
can be computed by the following rules:
\begin{lstlisting}
instate(activeProcess(B,K),T+1) :-  %\label{e4}
	play(action(_,_,B,_,K),T), time(T). %\label{e4a}
instate(activeProcess(B,K),T+1) :-  %\label{e5}
	not change(B,T+1),
	instate(activeProcess(B,K),T), time(T). %\label{e5a}
\end{lstlisting}
In other words, the state of step \texttt{T+1} contains one new active process $\texttt{B}_\texttt{K}$
resulting from the predicate \texttt{play(action(\_,\_,B,\_,K),T)} (lines \ref{e4}-\ref{e4a})
as well as all the unchanged processes that correspond to the other sorts (lines \ref{e5}-\ref{e5a}).

The overall result of the pieces of program presented in this subsection
is an answer set containing one answer for each
possible evolution in \texttt{n} time steps,
starting from a given initial state.

\subsection{Reachability verification}
In this section, we focus on the reachability of a set of processes which corresponds to the reachability property (see \pref{def:reachability}):
``Is it possible, starting from a given initial state, to play a number of actions so that a set of given processes are active in the resulting state?''
We now want to use the implementation of the dynamics computation of the previous section in order to solve this reachability problem.
For this, we first use a predicate \texttt{goal} to list the processes we want to reach and we add as many rules of the following form as there are objective processes:
\begin{lstlisting}
goal(activeProcess("z",1)). %\label{c1}
\end{lstlisting}
Then, the literal \texttt{reached(F, T)} 
checks if a given active process \texttt{F} of the goal
is contained in the state of step \texttt{T},
as defined in the rule of line \ref{c2}.
Else the answer will be eliminated by a constraint (not detailed here) which verifies if all processes of the goal are satisfied.
\begin{lstlisting}
reached(F, T) :-  goal(F), instate(F, T). %\label{c2}
\end{lstlisting}

However, the limitation of the method above is that the user has to decide upstream
the number of computed steps that should be sufficient to reach all the goals.
It is a main disadvantage like in \cite{roccaasp} because a search in $N$ steps will find no solution
if the shortest path to the goal requires $K$ steps with $K > N$.
It may also needlessly lengthen the resolution if the shortest path requires $n$ steps with $n << N$.
One solution is then to use an incremental computation mode,
which is especially tackled by the incremental solver of \textsc{Clingo}~\cite{gebser2008user}.
The corresponding syntax separates the program in 3 parts.
The \texttt{\#program base} part contains only non-incremental elements
and is thus used to declare general rules
that do not depend on the time steps (such as the data related to the model).
The body iteration is then written in the
\texttt{\#program step(t).} and \texttt{\#program check(t).} parts,
which are computed at each incremental step. The step number is not given by a variable but by a constant placeholder called "\texttt{t}'' in the following.
The first part comprises rules depending on the time step,
and the second contains constraints that stops the iteration when needed.

When using this new syntax, the obtained program is almost identical
to what was presented before,
except that step numbers \texttt{T}
are replaced by the constant placeholder \texttt{t}.
In each step \texttt{t}, the program computes: \\
  - the playable actions 
   \texttt{playable(Action,t)}, \\
   - the choosen action to be played \texttt{play(Action,t)},
  - the possible bounces \texttt{change(B,t)} \\
  - the new states \texttt{instate(activeProcess(A,I),t+1)}\\

in the \texttt{\#program step(t).} part
the same way than previously, but only for the current step.
The solver then compares its current answer sets with
the \texttt{t}-dependent constraint given in the \texttt{\#program check(t).} part.
Regarding our implementation, this constraint is given in line \ref{c4}
and simply states that all goals have to be met.
If this constraint invalidates all current answer sets,
the computation continues in the next iteration in order to reach a valid answer set.
As soon as some answer sets meet the constraints,
they are returned and the computation stops.
\begin{lstlisting}
notReached(t) :- goal(F), not instate(F,t). %\label{c3}
:- notReached(t), query(t). %\label{c4}
\end{lstlisting}


\section{Comparative performance analysis}
\label{sec:comparison}
In this section, we show the effectiveness of our approach on some examples,
and compare it to other existing approaches.
All computations (except the one called \textsc{ASP-Thomas}) were performed on a Pentium~V, 3.2~GHz with 4~GB RAM.

\subsection{Evaluation}
To assess the efficiency of our new approach,
we position ourselves with respect to existing methods dealing with different biological models.
We have chosen the following tools, that are detailed below: 
\textsc{GINsim}\footnote{\textsc{GINsim} version 2.4 alpha: \url{http://ginsim.org/}} (Gene Interaction Network Simulation)~\cite{gonzalez2006ginsim,naldi2009logical,naldi2007decision};
\textsc{LibDDD}\footnote{\textsc{LibDDD} version 1.8: \url{http://move.lip6.fr/software/DDD/}}
(Library of Data Decision Diagrams)~\cite{thierry2009hierarchical,colange2013towards};
\textsc{Pint}\footnote{\textsc{Pint} version 2015-11-14: \url{http://loicpauleve.name/pint/}}~\cite{PMR12-MSCS};
and a method for CTL model-checking proposed by Rocca \textit{et al.} in~\cite{roccaasp}
which was also developed in ASP.
%but for states transitions networks.
Each method uses a specific kind of representation\footnote{When available, we used the converters included into \textsc{Pint} for these translations.}:
Thomas models (a particular kind of logical regulatory networks) for \textsc{GINsim}
and the method of Rocca \textit{et al.},
instantiable transition systems for \textsc{LibDDD},
%state transition networks for the method of Rocca \textit{et al.}
and Process Hitting (PH) for \textsc{Pint} as well as for our method.

For this comparative study, we focus on biological network of different sizes:
a tadpole tail resorption (TTR) model with 12 biological components~\cite{khalis2009smbionet},
an ERBB receptor-regulated G1/S transition (ERBB) model with 20 components~\cite{Samaga2009}
and a T-cell receptor (TCR) signaling network of 40 components~\cite{Klamt06}.
These models were chosen to be of different sizes:
from small (12 components) to large (40 components).
We note however that the considered PH models may contain more sorts than
the original number of biological components, due to the use of
"cooperative sorts'', which allow to model Boolean gates but do not necessarily
have a biological meaning.
The different model representations that are required to perform these benchmarks have been obtained by translations
from the PH
ensuring the conservation of the dynamical properties.

The specification of the models and the results of our enumeration of the fixed points
are summed up in \pref{tab:models}.
The time performance is roughly the same than the SAT implementation
that comes with \textsc{Pint}.
The results for several methods regarding reachability properties
are summarized in \pref{tab:reachability}.
The methods and the results provided by each of them are detailed in the following.
The overall results show that our method is efficient in computing reachability
from a given initial state;
furthermore, it sometimes provides more information than the other existing ones.

\begin{table*}[ht]
\begin{center}
\noindent%
\begin{tabular}{|l|c||c|c|c||>{\columncolor{verylightgray}}c|>{\columncolor{verylightgray}}c|}
\hline
  \multicolumn{2}{|c||}{Models} & \multicolumn{3}{c||}{PH representation} & \multicolumn{2}{c|}{Fixed points enumeration} \\
\hline
  Name & Components & Sorts & Processes & States & computation time & Nbr of results \\
\hline
\hline
  TTR  & 8  & 12 & 42  & $2^{19}$ & 0.004s & 0 \\
\hline
  ERBB & 20 & 42 & 152 & $2^{70}$ & 0.017s & 3 \\
\hline
  TCR  & 40 & 54 & 156 & $2^{73}$ & 0.021s & 1 \\
\hline
\end{tabular}
\vspace*{4pt}
\caption{\label{tab:models}%
Description of the models used in our tests and results of our fixed point enumeration.
Each model is referred to by its short name, where
TTR stands for the tadpole tail resorption model~\cite{khalis2009smbionet},
ERBB for the receptor-regulated G1/S transition of the same name~\cite{Samaga2009}
and TCR for the T-cell receptor signaling network~\cite{Klamt06}.
For each of them, this table gives the number of biological components
in the original representation,
and the number of sorts, the number of processes
and the number of states in the PH model.
Finally, the last column gives the computation time for the enumeration of all fixed points
and the number of results returned.
}
\end{center}
\end{table*}

\begin{table*}[ht]
\begin{center}
\noindent%
\begin{tabular}{|l|l|c||c|c|c|>{\columncolor{verylightgray}}c|}
\hline
  \multicolumn{3}{|c||}{Experiments} & \multicolumn{4}{c|}{Results} \\
\hline
  & Model & Target  & \textsc{Pint} & \textsc{LibDDD} & \textsc{GINsim} & \textsc{ASP-PH} \\
\hline
\hline
  \#1 & TTR & full state & 0m0.97s & 0m1.15s &  0m2.05s & 0m1.90s \\
\hline
  \#2 & ERBB & full state & out &1m55.38s & 2m31.64s & 0m11.84s \\
\hline
  \#3 & ERBB & partial state  & 0m0.03s &1m54.96s & -- & 0m5.02s \\
\hline
  \#4 & TCR & full state & Inconc & out & out & 6m27.93s \\
\hline
  \#5 & TCR & partial state & 0m0.02s & out & -- & 1m35.08s \\
\hline
\end{tabular}
\vspace*{4pt}
\caption{\label{tab:reachability}
Compared performances of several methods to compute reachability analyses:
The method of \textsc{Pint}, \textsc{LibDDD}, \textsc{GINsim} and our new method presented in this paper, called \textsc{ASP-PH}.
For each test, this table gives the short name of the considered model,
as given in table~\ref{tab:models},
the type of goal (either a whole state or a sub-state)
and the computation time of the different methods used for the tests,
where ``out'' marks an execution taking too much time or memory,
``\mbox{--}'' indicates that is not possible to do the test,
and ``Inconc'' states that the method terminates without a response.
}
\end{center}
\end{table*}

% Tableau qualitatif
% Bound
% Loop detection
% Search:

\begin{itemize}[leftmargin=*]

\item \textbf{\textsc{GINsim}} is a software for the edition, simulation and analysis
of gene interaction networks.
It allows to compute all reachable fixed points from a given initial state instantly;
however, it is not possible to compute all fixed points independently from the initial state.
Regarding the reachability problem, \textsc{GINsim} only allows to check the reachability of
full states, because its approach consists in computing
(part of) the state-transition graph and then searching for a path between the two given states.
Therefore, it was not possible to perform reachability checks on partial states
(experiments \#3 \& \#5).
Small state-transition graphs can also be displayed by this tool.

\item \textbf{\textsc{LibDDD}}
is a library for symbolic model-checking of CTL \& LTL properties.
It can thus especially be used to check reachability properties;
however, as opposed to our method, it does not output an execution path
solving this reachability.
In addition, it relies on the construction of the state-transition graph
which is then stored under the form of a binary decision diagram for a more efficient analysis.
This computation explains why \textsc{LibDDD} takes more time to respond,
and gets out of memory in about 12 minutes for the biggest example
which contains $2^{73}$ states
(experiments \#4 \& \#5).
Finally, \textsc{LibDDD} is not able to compute the fixed points of a network.

\item \textbf{\textsc{Pint}}
is a library gathering tools and converters related to the PH.
It should be noted that \textsc{Pint} contains the only reachability analysis
developed so far natively for the Process Hitting,
before the method proposed in this paper.
It consists in an approximation that avoids to compute the state-transition graph;
it is thus ensured to be really efficient, which explains the fastest results,
but at the cost of possibly terminating without being conclusive.
However, it is not designed for goals consisting of many processes,
which are more likely to trigger an inconclusive response
(such as for experiment \#4),
or an exponential research in sub-solutions
(such as for experiment \#2).
This explains the high computation times for some of the tests.
Moreover, in the case of a positive answer,
it currently does not return the execution of the path achieving the desired reachability,
but only outputs its conclusion (True, False or Inconclusive).

\item \textbf{\textsc{ASP-Thomas}}\footnote{The authors wish to thank Laurent Trilling for his help to perform the tests.}
offers the possibility to model-check CTL properties of Thomas networks. 
There is however no automatic way that allows the modeling of Thomas networks in ASP, which currently has to be made by hand and requires labeling. As for our method, we use the PH whose actions are easily represented in ASP with one fact per action.
In addition, the method \textsc{ASP-Thomas} requires to provide a maximum number of steps
for which the dynamics will be computed, which may be difficult to be predicted. Furthermore, all dynamics of the system have to be computed up to $n$ steps in order to perform a depth first verification.
It is not the case for our method which can stop as soon as the goal is met, and thus potentially before the $n$\textsuperscript{th} step.
%, the program will stop at the first step that reach the goals: breath first search.
However it is clear that with a minimal number of steps this approach terminates very quickly when compared with others.
It also shows that ASP is a good choice to run the dynamics of a model and check reachability properties.
The a summary of differences between our approach and \textsc{ASP-Thomas} is detailed in \pref{tab:qualitative_differences}.
The main difference is that, in \textsc{ASP-Thomas}, the search is bounded.
If the number of steps to reach the goal is part of the goal or if one can a priori know the length of the path to reach the specified goal,
this approach can provide very good performance.
But if the bound is totally unknown, choosing a too small path length can naturally lead to miss the goal.
%and thus to conclude it is unreachable when it is reachable.
On the other hand, choosing a bound that is too big will greatly impact the performance.
For example, in experiment \#3, the goal is reachable with a path of 18 steps,
and using a bound of 21, \textsc{ASP-Thomas} finishes in 2.61s.
However, if we fix the bound to 30 steps, it takes several minutes.
We can observe similar results on the others benchmarks.
With our approach, if the goal is reachable, the run time only depends on the distance to reach the goal.
In \textsc{ASP-Thomas}, the run time only depends of the chosen bound, because all paths of the chosen length will be generated before being checked.

\end{itemize}

\subsection{Strengths and limitations of our method}
\label{sec:limitations}

In the previous sections,
we developed two new methods in ASP to check dynamical properties,
namely identifying fixed points and finding all the shortest paths to reach a given goal.
Compared to some other methods described above
(\textsc{GINsim} and \textsc{LibDDD}) our method is relatively faster and also permits to study larger networks
(up to $2^{73}$ states in our tests). We exclusively studied networks modeled in Process Hitting. This new formalism for network modeling is a restriction of synchronous automata and thus allows to represent any kind of dynamical model. Moreover, the dynamics of PH it is easy to implement into ASP.

In practice, we can detect the loops in the model dynamics and avoid to check the same path again.
However, the iterative version of \textsc{Clingo} will continue to iterate until the goal is reached,
even is stuck in a loop, while it should return an \texttt{unsatisfiable} response.
This is due to the fact that between each step, all information about the previous visited states is lost.
To our knowledge, there is no possibility yet to force the incrementation to stop in this iterative \textsc{Clingo} version.
It is still possible, however, to limit the number of iterations to an arbitrary
maximum which will be eventually reached in such case.
This is possible with the use of a construct limiting the maximum number of steps: ``\texttt{\#const imax = n.}'',
where \texttt{n} is the maximum number of steps.

Several values can be given to this parameter \texttt{n}.
For example, the total number of states is an obvious maximum,
as it will never be exceeded by a minimum path,
but it is too hight to be very interesting.
The total number of sorts is a more interesting value,
under the hypothesis that each one will change its active process at most once,
which is often the case for Boolean networks;
or, with a similar reasoning, the total number of processes can be chosen.

Given our implementation, if the step \texttt{n} is reached (meaning no valid path was found),
the computation stops with an \texttt{unsatisfiable} response for the reachability.
However, if the goal is reached at some intermediary step, it is not necessary to continue the iteration up to ``\texttt{n}''.
Instead, the computation is stopped and the current trace of the reachability is returned.
In comparison, \textsc{ASP-Thomas} has to compute all possible dynamic evolutions of length ``\texttt{n}'' before verifying the property.

\begin{table}[ht]
\begin{center}
\noindent%
\begin{tabular}{|l|l|c|}
\hline
  & \textsc{ASP-Thomas} & \textsc{ASP-PH} \\
\hline
\hline
 Input Model & Thomas network & Process hitting \\
\hline
 Input generation & Hand-made & {\bf Automatic} \\ %with PH-to-ASP \\
\hline
 Reachability & Bounded & {\bf Unbounded} \\
\hline
 Non-reachability & Bounded & Bounded \\
\hline
 Search & Depth first & Breadth first \\
  \hline
 Exec. time: small imax & seconds/minutes & seconds/minutes \\
 \hline
 Exec. time: big imax & hours/out &seconds/minutes  \\
\hline
\end{tabular}
\vspace*{4pt}
\caption{\label{tab:qualitative_differences}
Qualitative comparison between our approach and \textsc{ASP-Thomas}.
}
\end{center}
\end{table}




%\vspace{-1.0cm}

\section{Conclusion and future directions}
\label{sec:ccl}
In this paper, we proposed a new logical approach to address dynamical properties of Process Hitting models. The originality of our work consists in using ASP a powerful declarative programming paradigm. Thanks to the encoding we introduced, we are not only able to tackle the enumation of fixed points but also to consider reachability properties. The major benefit of such a method is to get an exhaustive enumeration of all corresponding paths while still being tractable for models with dozens of interacting components.

The implementation of these problems was given into ASP,
and applied to several biological examples of various sizes, up to
40 biological components.
The results showed that our implementation is faster and deals with bigger models
than other classical approaches, especially \textsc{LibDDD} which is a symbolic model-checker.

Our work could benefit from several extensions, 
%starting by discussing if there is possible improvement to the solver of the iterative version of \textsc{Clingo} to treat the final infinite loop in some cases with unreachable goals -- and thus also the need to arbitrarily cap the number of iterations.
starting by direct improvements to the developed tool,
such as removing the final infinite loop in some cases when the goal is not reachable --
and thus also the need to arbitrarily cap the number of iterations. We think also that possible improvements to the solver of the iterative version of \textsc{Clingo} could be necessary to treat this unsatisafaible cases.
Of course, the set of applicable models can also be extended,
for example with the addition
of priorities or neutralizing edges,
or by considering synchronous dynamics or other representations
such as Thomas modeling~\cite{BernotSemBRN}.
However, the range of the analysis can also be extended,
by searching instead the set of initial states
allowing to reach a given goal,
or extending the method to universal properties (like the $\mathsf{AF}$ operator).






% conference papers do not normally have an appendix


% use section* for acknowledgement
\section*{Acknowledgment}

The European Research Council has provided financial support
under the European Community's Seventh Framework Programme (FP7/2007--2013)~/
ERC grant agreement no.~259267.


\bibliographystyle{./IEEEtran}
\bibliography{./biblio}


\end{document}


