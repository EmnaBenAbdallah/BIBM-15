The study of fixed points (and, more generally, basins of attraction) provides an important understanding of the different behaviors of a Biological Regulatory Network (BRN)~\cite{wuensche1998genomic}.
Indeed, a system will always eventually end in a basin of attraction,
and this may depend on biological switch or other complex phenomena.
A fixed point is a state of the BRN in which it is not possible to have any new dynamic evolutions anymore;
in other words, it is a basin of attraction that is composed of only one state.

In the following, we consider a Process Hitting $\PH = (\Sigma, \PHl, \PHh)$.
A state $s \in L$ is a fixed point (or stable state) of the Process Hitting model if and only if it has no successor state, \ie regarding the active processes in $s$ there is no playable action (see \pref{def:fixpoint}).
Therefore, it has been shown~\cite{PMR10-TCSB} that
a stable state in a Process Hitting network is a state so that
every active process does not hit or is not hit by another process in the same state.
We note especially that given this result, processes involved in a self-hit (an action whose hitter and target processes are the same) cannot be part of a stable state.

\subsection{Answer Set Programming}

In this section, we recapitulate the basic elements of ASP \cite{baral2003knowledge, Vladimir, Glimpse}.
An answer set program is a finite set of rules of the form
  \[a_{0}\ \leftarrow \ a_{1},\ \ldots,\ a_{m},\ not\ a_{m+1},\ \ldots,\ not\ a_{n}\]
where $n \ge m \ge 0$, $a_{0}$ is a propositional atom or $\bot$, all
$a_{1}, \ldots ,a_{n}$ are propositional atoms and the symbol ``$not$'' denotes default negation.
The intuitive reading of such a rule is that whenever $a_{1}, \ldots, a_{m}$
are known to be true and there is no evidence for any of the default negated atoms $a_{m+1}, \ldots, a_{n}$ to be true, then $a_{0}$ has to be true as well.
If $a_{0} = \bot$, then the rule becomes a constraint (in which case $a_{0}$ is usually omitted).
As $\bot$ can never become true, if the right-hand side of a constraint is validated, it invalidates the whole answer set.

In the ASP paradigm, the search of solutions consists in computing the \emph{answer sets} of a given answer set program.
An answer set for a program is defined by Gelfond and Lifschitz \cite{DBLP:conf/iclp/GelfondL88} as follows.
An \emph{interpretation} $I$ is a finite set of propositional atoms.
%An atom $a_i$ is \emph{true under $I$} if $a_i \in I$, and false otherwise.
A rule $r$ as given above is \emph{true under~$I$} if and only if:
  \[\{a_1,\ \dots,\ a_{m}\} \subseteq I \wedge \{a_{m+1},\ \ldots,\ a_{n}\} \cap I = \emptyset \Rightarrow a_{0} \in I \enspace.\]
An interpretation $I$ is a \emph{model} of a program $P$ if each rule $r \in P$ is true under $I$.
Finally, $I$ is an answer set of $P$ if $I$ is a minimal (in terms of inclusion) model of $P^{I}$,
where $P^{I}$ is defined as the program that results from $P$ by deleting all rules that contain a default negated atom from $I$,
and deleting all default negated atoms from the remaining rules.
Programs can yield no answer set, one answer set, or many answer sets.
To compute answer sets of an answer set program, one needs a grounder (to remove variables from the rules) and a solver.
For the present work, we used \textsc{Clingo}\footnote{We used \textsc{Clingo} version 4.5.0: \url{http://potassco.sourceforge.net/}}~\cite{gebser2008user} which is a combination of both.

\subsection{Process Hitting translation in ASP}
Before analyzing the dynamics of the network,
we first need to translate the concerned PH network into ASP\footnote{All programs, including this translation and the methods described in the following, are available online at: \url{https://github.com/EmnaBenAbdallah/verification-of-dynamical-properties_PH}}.
To do this we use the following self-describing predicates:
\texttt{sort} to define sorts, \texttt{process} for the processes and \texttt{action} for the network actions.
\pref{ex:asp-ph} shows how a PH network is defined with these predicates.

\begin{example}[Representation of a PH network in ASP]
\label{ex:asp-ph}
The representation of the PH network of \pref{fig:ph} in ASP is the following:
\begin{lstlisting}
process("a", 0..1). process("b", 0..1). %\label{ASPprocess1}
process("z", 0..1). process("ab", 0..3). %\label{ASPprocess2}
sort(X) :- process(X,_). %\label{ASPsort}
action("a",0,"b",0,1). action("b",1,"b",1,0). %\label{actions1}
action("b",0,"ab",1,0). action("b",0,"ab",3,2). %\label{actions2}
action("b",1,"ab",0,1). action("b",0,"ab",2,3). %\label{actions3}
action("a",0,"ab",2,0). action("a",0,"ab",3,1). %\label{actions4}
action("a",1,"ab",0,2). action("a",1,"ab",1,3). %\label{actions5}
action("ab",3,"z",0,1). %\label{actions6}
\end{lstlisting}
In \refll{ASPprocess1}{ASPprocess2} we create the list of processes (expression level) corresponding to each sort,
for example the sort "\texttt{a}'' has 2 processes numbered \texttt{0} and \texttt{1};
this predicate will in fact expand into the two following predicates:
\begin{lstlisting}[numbers=none]
process("a", 0). process("a", 1).
\end{lstlisting}
The processes of the cooperative sort ``\texttt{ab}'',
which represents a cooperation between the biological components ``\texttt{a}'' and ``\texttt{b}'',
have been relabeled \texttt{0}, \texttt{1}, \texttt{2} and \texttt{3}.
\Refl{ASPsort} enumerates every sort of the network from the previous information.
In ASP, a word starting with a capital letter is a variable,
and the underscore (``\texttt{\_}'') is a placeholder for any value.
Finally, all the actions of the network are defined straightforwardly in \refll{actions1}{actions6};
for instance, the predicate \texttt{action("a",0,"b",0,1)} represents the action
$\PHfrappe{a_0}{b_0}{b_1}$.
\end{example}

\subsection{Search of fixed points}

The enumeration of fixed points requires to translate the definition of a stable state (see \pref{def:fixpoint})
into an ASP program.
The first step consists of eliminating all processes involved in a self-hit;
the other processes are recorded by the predicate \texttt{shownProcess} (\refll{hiddenProcess}{shownProcess2}).
\begin{lstlisting}
hiddenProcess(A,I) :- action(A,I,B,J,K), A=B. %\label{hiddenProcess}
shownProcess(A,I) :- not hiddenProcess(A,I), %\label{shownProcess1}
                     process(A,I). %\label{shownProcess2}
\end{lstlisting}
Then, we have to browse all remaining processes of this graph in order to generate all possible states,
that is, all possible combinations of processes by choosing only one process from each sort (\refll{select-processes1}{select-processes2}).
%So that \refll{hiddenProcess}{shownProcess2} is an optimization to reduce the number of possible states.
\begin{lstlisting}
1 { selectedProcess(A,I) : shownProcess(A,I) } 1 %\label{select-processes1}
         :- sort(A). %\label{select-processes2}
\end{lstlisting}
The previous cardinality rule creates as many potential answer sets as there are possible states
to take into account.
Finally, the last step consists in filtering out any state that is not a fixed point,
or, in other words, eliminating all candidate answer sets in which an action could be played (and thus lead the network into another state).
For this, we use a constraint:
any solution that satisfy the body of this constraint will be removed from the answer set.
Regarding our problem, a state is eliminated if there exists an action between two selected processes (\refll{constraintFix1}{constraintFix}).
\begin{lstlisting}
:- action(A,I,B,J,_), selectedProcess(A,I), %\label{constraintFix1}
    selectedProcess(B,J), A!=B. %\label{constraintFix}
\end{lstlisting}
In the end, the fixed points of the considered model are exactly the states represented in each remaining answer,
described by the atoms \texttt{fixProc(A,I)} (\refl{fixproc}).
% Finally the \texttt{selectedProcess(A,I)} permits to define the fixed point which is compound by these selected processes:
\begin{lstlisting}
fixProc(A,I) :- selectedProcess(A,I). %\label{fixproc}
\end{lstlisting}

\begin{example}[Fixed points enumeration]
The PH model of \pref{fig:ph} contains 4 sorts:
$a$, $b$ and $z$ have 2 processes and $ab$ has 4; therefore, the whole model has $2*2*2*4 = 32$ states (whether they can be reached or not from a given initial state).
We can check that this model contains $2$ fixed points: $\PHstate{b_0, a_1, ab_2, z_0}$ and $\PHstate{b_0, a_1, ab_2, z_1}$.
Indeed, there is no action between each two processes contained in this state so no execution is possible from these.
In this example, no other states verify this property.

If we execute the ASP program detailed below (see \refll{hiddenProcess}{constraintFix}),
alongside with the description of the PH model given in \pref{ex:asp-ph} (see \refll{ASPprocess1}{actions5}),
we obtain two answer sets that match the expected result:
\begin{lstlisting}[numbers=none]
Answer 1 : fixProc(a,1), fixProc(b,0), 
		fixProc(z,0), fixProc(ab,2)
Answer 2 : fixProc(a,1), fixProc(b,0), 
		fixProc(z,1), fixProc(ab,2)
\end{lstlisting}
\end{example}