\label{sec:prem-def}

\pref{def:PH} introduces the Process Hitting~(PH)~\cite{PMR10-TCSB}
which allows to model a finite number of local levels,
called \emph{processes},
grouped into a finite set of components, called \emph{sorts}.
A process is noted $a_i$, where $a$ is the sort's name,
and $i$ is the process identifier within sort $a$.
At any time, exactly one process of each sort is \emph{active},
and the set of active processes is called a \emph{state}.

The concurrent interactions between processes are defined by a set of \emph{actions}.
Each action is responsible for the replacement of one process by another of the same sort
conditioned by the presence of at most one other process in the current state.
An action is denoted by $\PHfrappe{a_i}{b_j}{b_k}$, which is read as
``$a_i$ \emph{hits} $b_j$ to make it \emph{bounce} to $b_k$'',
where $a_i$, $b_j$, $b_k$ are processes of sorts $a$ and $b$,
called respectively \emph{hitter}, \emph{target} and
\emph{bounce} of the action.
We also call a \emph{self-hit} any action whose hitter and target sorts are the same,
that is, of the form: $\PHfrappe{a_i}{a_i}{a_k}$.

The PH is therefore a restriction of synchronous automata, where each transition
changes the local state of exactly one automaton,
and is triggered by the local states of at most two distinct automata.
This restriction in the form of the actions was chosen to permit
the development of efficient static analysis methods
based on abstract interpretation~\cite{PMR12-MSCS} as well as the dynamic analysis that we present in this paper. In addition due to this restriction we need to add non biological components into the system that allow to present the cooperation between $2$ or more components which we call \emph{cooperative sorts}. So it maintains the effectiveness of the static and dynamic analysis.

\begin{definition}[Process Hitting]\label{def:PH}
  A \emph{Process Hitting} is a triple $(\PHs,\PHl,\PHa)$ where:
  \begin{itemize}
    \item  $\PHs = \{a,b,\dots\}$ is the finite set of \emph{sorts};
    \item  $\PHl = \prod_{a\in\PHs} \PHl_a$ is the set of \emph{states} where
      $\PHl_a = \{a_0,\dots,a_{l_a}\}$
      is the finite set of \emph{processes} of sort $a\in\Sigma$
      and $l_a$ is a positive integer, with $a\neq b\Rightarrow \PHl_a \cap \PHl_b = \emptyset$;
    \item  $\PHa = \{ \PHfrappe{a_i}{b_j}{b_k} \in \PHl_a \times \PHl_b^2 \mid
      (a,b) \in \PHs^2 \wedge b_j\neq b_k \wedge a=b\Rightarrow a_i=b_j\}$
      is the finite set of \emph{actions}.
  \end{itemize}
\end{definition}

\begin{example}

\tikzstyle{prio}=[draw,thick,-stealth]

\begin{figure}[ht]
\pref{fig:ph} represents a PH $(\PHs,\PHl,\PHa)$ with four sorts
  \centering
  \scalebox{1.4}{
  \begin{tikzpicture}
    \TSort{(0,0)}{a}{2}{b}
    \TSort{(0,4)}{b}{2}{t}
    \TSort{(5,0.5)}{z}{2}{r}

    \TSetTick{ab}{0}{00}
    \TSetTick{ab}{1}{01}
    \TSetTick{ab}{2}{10}
    \TSetTick{ab}{3}{11}
    \TSort{(0.5,2)}{ab}{4}{t}

    \THit{a_0}{}{ab_3}{.south west}{ab_1}
    \THit{a_0}{}{ab_2}{.south west}{ab_0}
    \THit{a_1}{}{ab_1}{.south}{ab_3}
    \THit{a_1}{}{ab_0}{.south}{ab_2}

    \THit{b_0}{}{ab_3}{.north}{ab_2}
    \THit{b_0}{}{ab_1}{.north}{ab_0}
    \THit{b_1}{}{ab_2}{.north}{ab_3}
    \THit{b_1}{}{ab_0}{.north}{ab_1}
    
    \THit{b_1}{selfhit}{b_1}{.north west}{b_0}
    \THit{a_0}{bend left}{b_0}{.south west}{b_1}

    \THit{ab_3}{}{z_0}{.west}{z_1}

    \path[bounce, bend right]
     \TBounce{ab_1}{}{ab_3}{.south}
     \TBounce{ab_0}{}{ab_2}{.south}
     \TBounce{ab_3}{}{ab_2}{.east}
     \TBounce{ab_1}{}{ab_0}{.east}
    ;
    \path[bounce, bend left]
      \TBounce{ab_3}{}{ab_1}{.south east}
      \TBounce{ab_2}{}{ab_0}{.south east}
      \TBounce{ab_2}{}{ab_3}{.west}
      \TBounce{ab_0}{}{ab_1}{.west}
    ;
    \path[bounce, bend right]
      \TBounce{b_1}{}{b_0}{.north east}
      \TBounce{b_0}{}{b_1}{.south west}
    ;
    \path[bounce, bend left]
      \TBounce{z_0}{}{z_1}{.south west}
    ;
    \TState{a_0, b_0, ab_0, z_0}
  \end{tikzpicture}
  }
  \caption{\label{fig:ph}
A PH model example with four sorts: $a$, $b$, $ab$ and $z$. Boxes represent the sorts (network components), circles represent the processes (component levels), and the actions that model the dynamic behavior are depicted by pairs of arrows in solid and dotted lines. $a$ is either at level 0 or 1, $b$ at either level 0 or 1, $z$ at either level 0, 1 or 2 and the cooperative sort $ab$ has 4 levels corresponding to the combinaison of the levels of the corresponding sorts $a$ and $b$. The grayed processes stand for the possible initial state $\PHstate{a_0, b_0, ab_{00}, z_0}$.
  }
\end{figure}


\end{example}

A state of the networks is a set of active processes containing a single process of each sort.
The active process of a given sort $a \in \PHs$ in a state $s \in \PHl$
is noted $\PHget{s}{a}$.
For any given process $a_i$ we also note: $a_i \in s$ if and only if $\PHget{s}{a} = a_i$. It means that the component, or the \emph{sort}, $a$ is at level $i$ during the state $s$.

\begin{definition} [Playable action]
\label{def:playableAction}
Let $\PH = (\PHs,\PHl,\PHa)$ be a Process Hitting and $s \in \PHl$ a state of $PH$.
We say that the action $h = \PHfrappe{a_i}{b_j}{b_k} \in \PHa$
is \emph{playable in state $s$} if and only if
$a_i \in s$ and $b_j \in s$ (\ie $\PHget{s}{a} = a_i$ and $\PHget{s}{b}=b_j$).
The resulting state after playing $h$ in $s$
is called a \emph{successor} of $s$ and
is denoted by $(s \play h)$,
where $\PHget{(s \play h)}{b} = b_k$ and
$\forall c \in \PHs, c \neq b \Rightarrow \PHget{(s \play h)}{c}=\PHget{s}{c}$.
\end{definition}

PH was chosen for several reasons. It is a general framework that was mainly used for biological networks. Besides, it allows to represent any kind of dynamical model,
and converters to several other representations are available and included into \textsc{Pint}\footnote{\textsc{Pint} version 2015-11-14: \url{http://loicpauleve.name/pint/}}~\cite{PMR12-MSCS}. Espacially for the particular form of the actions in a PH model allows
to easily represent them using the Answer Set Programming \cite{Baral03, Vladimir, Glimpse, sureshkumar2006ansprolog},
with one fact per action. We note that although an efficient dynamical analysis already exists for this framework,
based on an approximation of the dynamics,
it is interesting to identify its limits
and compare them to the exhaustive approach we present in this paper.

\subsection{Dynamical properties}

The study of the dynamics of biological networks was the focus of many works, explaining the diversity of network modelings and the different methods developed in order to check dynamic properties.
In this paper we focus on 2 main properties: the stable states and the reachability.
In the following, we consider a PH model $(\PHs,\PHl,\PHa)$,
and we formally define these properties
and explain how they could be verified on such a network.

The notion of \emph{fixed point}, also called \emph{stable state},
is given in \pref{def:fixpoint}.
A fixed point is a state which has no successor.
Such states have a particular interest as they denote states in which the model
stays indefinitely,
and the existence of several of these states denotes a switch in the dynamics~\cite{wuensche1998genomic}.

\begin{definition}[Fixed point]
\label{def:fixpoint}
  A state $s \in \PHl$ is called a \emph{fixed point}
  (or equivalently \emph{stable state})
  if and only if it has no successors.
  In other words, and in $PH$ symantic, $s$ is a fixed point if and only if no action is playable in this state:
  \[\forall \PHfrappe{a_i}{b_j}{b_k} \in \PHa, a_i \notin s \vee b_j \notin s \enspace.\]
\end{definition}

A finer and more interesting dynamical property consists in
the notion of \emph{reachability}.
Such a property, defined in \pref{def:reachability},
states that starting from a given initial state, it is possible
to reach a given goal, that is, a state that contains a process
or a set of processes.
Checking such a dynamical property is considered difficult
as in usual model-checking techniques,
it is required to build (a part of) the state graph,
which has an exponential complexity.

In the following, if $s \in \PHl$ is a state,
we call \emph{scenario in $s$}
any sequence of actions that are successively playable in $s$.
We also note $\Sce(s)$ the set of all scenarios in $s$.
Moreover, we denote by $\Proc = \bigcup_{a \in \PHs} \PHl_a$
the set of all process in $\PH$.

\begin{definition}[Reachability property]
\label{def:reachability}
  If $s \in \PHl$ is a state and $A \subseteq \Proc$ is a set of processes,
  we denote by $\mathcal{P}(s, A)$ the following \emph{reachability property}:
  \[\mathcal{P}(s, A) \equiv \exists \delta \in \Sce(s), \forall a_i \in A, \PHget{(s \play \delta)}{a} = a_i
    \enspace.\]
\end{definition}

The rest of the paper focuses on the resolution of the previous issues. We demonstrate our algorithms about: the simulation of a biological regulatory network modeled in PH,
the enumeration of all fixed points (\pref{sec:fixpoint} and the verification of a reachability property will (\pref{sec:dynamics}).
